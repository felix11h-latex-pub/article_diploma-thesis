

\section{Discussion}\label{sec:discussion}



% The structural analysis of anisotropic networks revealed a network
% topology characteristically different from comparable
% distance-dependent or even rewired networks. This however only, higher
% order connectivity motifs .

% Vertex degree distributions were found to have increased variance and
% skew. However, as rewired and in the case of in-degrees even
% distance-dependent networks also show this feature, increased variance
% and skew are only indirectly caused by network anisotropy. Similarly,
% while an\-iso\-tro\-pic networks display a small-world property, the
% short path length and high clustering coefficient are not due to
% anisotropy but stem from the network's distance-dependency. In fact,
% eliminating anisotropy in the network can lead to even shorter path
% lengths and higher clustering coefficients and thus strengthening the
% small-worldness of the network. Finally, anisotropy is not able to
% explain overrepresentation reciprocal connections reported repeatedly
% to be present in cortical networks.

% In terms of higher order connectivity however, anisotropy yields
% highly interesting results. Three-neuron motifs occurrences are not
% only distinct, but resemble in many aspects the profiles found by
% \textcite{Song2005}. 


 
% In their study \textcite{Perin2011} were able to relate increased
% edge-counts in neuron clusters to a \textit{common neighbor rule} - a
% specific relationship between common in- or outputs in a neuron pair
% and the probability for the pair to be connected - identifying an
% underlying principle for the occurrence of non-random connectivity in
% local circuits. Perin et al.\ suggest that (fire together wire
% together). In this structural analysis we found non-random
% connectivity resembling the findings of Song et al./ and , purely from
% a g. Identifying a common neighbor rule as the inherent, that
% non-random connectivity statistics might arrise from morphological
% restrictions rather than plastic processes.




% Taking distance-dependent networks as a reference, Perin et al.\
% were able to show network connectivity features in cortical circuits
% going beyond distance-dependency. A network model hoping to reflect
% such features, clearly is required to implement connectivity going
% beyond distance-dependent connection probabilities. With an




% Thus, anisotropy in connectivity seems . Initially reporting ,i
%  Perin
% et al.\ identify connectivity features beyond distance . Only a
% network model implementing connectivity rules going beyond
% distance-dependency can possible repo. 

% Introducing such a concept motivated
% from neuronal morphology, we here anisotropy in connectivity
% as an underlying connectivity principle 


% We find that anisotropy in connectivity induces increased high edge
% counts in neuron clusters, matching overrepresentations identified in
% local cortical circuits. 

% Comparing however to distance-dependent as opposed to rewired
% networks, we find that overrepresentation is stronger
% (\autoref{fig:perin_rew_dist}).


% Anisotropic networks inherently carry a connection principle that goes
% beyou




%%% Local Variables: 
%%% mode: latex
%%% TeX-master: "../dplths_document"
%%% End: 
