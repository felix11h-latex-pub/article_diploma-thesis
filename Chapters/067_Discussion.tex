

\section{Discussion}\label{sec:discussion}



In their study \textcite{Perin2011} were able to relate increased
edge-counts in neuron clusters to a \textit{common neighbor rule} - a
specific relationship between common in- or outputs in a neuron pair
and the probability for the pair to be connected - identifying an
underlying principle for the occurrence of non-random connectivity in
local circuits. Perin et al.\ suggest that (fire together wire
together). In this structural analysis we found non-random
connectivity resembling the findings of Song et al./ and , purely from
a g. Identifying a common neighbor rule as the inherent, that
non-random connectivity statistics might arrise from morphological
restrictions rather than plastic processes.




% Taking distance-dependent networks as a reference, Perin et al.\
% were able to show network connectivity features in cortical circuits
% going beyond distance-dependency. A network model hoping to reflect
% such features, clearly is required to implement connectivity going
% beyond distance-dependent connection probabilities. With an




% Thus, anisotropy in connectivity seems . Initially reporting ,i
%  Perin
% et al.\ identify connectivity features beyond distance . Only a
% network model implementing connectivity rules going beyond
% distance-dependency can possible repo. 

% Introducing such a concept motivated
% from neuronal morphology, we here anisotropy in connectivity
% as an underlying connectivity principle 


% We find that anisotropy in connectivity induces increased high edge
% counts in neuron clusters, matching overrepresentations identified in
% local cortical circuits. 

% Comparing however to distance-dependent as opposed to rewired
% networks, we find that overrepresentation is stronger
% (\autoref{fig:perin_rew_dist}).


% Anisotropic networks inherently carry a connection principle that goes
% beyou




%%% Local Variables: 
%%% mode: latex
%%% TeX-master: "../dplths_document"
%%% End: 
