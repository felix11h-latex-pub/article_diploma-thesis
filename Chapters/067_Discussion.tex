

\section{Discussion}\label{sec:discussion}

The structural analysis of anisotropic networks revealed connectivity
characteristically different from comparable distance-dependent or
even rewired networks. Such differences however only occur when
considering patterns in connectivity of three or more neurons, pair
connections and standard network measures remain unaffected.

Accordingly, vertex degree distributions in anisotropic networks were
found to have increased variance and skew, but as rewired and
distance-dependent networks also show this feature, this is only
indirectly caused by network anisotropy. Similarly, while
an\-iso\-tro\-pic networks display a small-world property, short path
lengths and high clustering coefficients are not due to anisotropy,
but stem from the network's distance-dependency. In fact we find that
eliminating anisotropy in the network can lead to even shorter path
lengths and higher clustering, strengthening the small-worldness of
the network. Finally, we were able to present conclusive evidence that
anisotropy is not able to reproduce overrepresentation of reciprocally
connected neuron pairs, reported repeatedly in cortical networks.

In terms of higher order connectivity however, anisotropy yields
highly interesting results. Three-neuron motif occurrences are not
only strongly influenced by anisotropy, but resulting
overrepresentations resemble in many aspects the profiles found by
\textcite{Song2005}. Furthermore, anisotropy in connectivity induces
an increased frequency of high edge counts in neuron clusters,
matching results from experiments in the rat's
cortex \parencite{Perin2011}. In their study, Perin et al.\ were able
to relate such edge count overrepresentations to a \textit{common
  neighbor rule} - a specific relationship between shared in- or
outputs of a neuron pair and the probability of connection in the
pair. Being able to reproduce increased high edge counts by imposing
such relation distance-dependent networks, the common neighbor rule
was identified as an underlying organization principle. By the
interpretation of the authors, stereotypically connected clusters
arising from this rule then constitute \enquote{building blocks} of
cortical networks to be molded by experience.

In this study, anisotropy was recruited as a connection principle for
the generation of geometric networks, reflecting the dimensions of
typical cortical slices. Capturing stereotypical axonal and dendritic
morphology of pyramidal cells in the cortex, those anisotropic
networks inherently feature connectivity going beyond
distance-dependency. Indeed, not only do anisotropic networks display
non-random patterns as found in local cortical circuits, but we were
here able to show that a common neighbor rule itself is highly
affected by anisotropy. Together, these findings promote the concept
of anisotropy as an important underlying connection principle.

The work presented here therefore provides new insight in how
non-random connectivity in cortical circuits may arise from
morphological characteristics inherent to cells in cortical
networks. Employing an abstract model capturing stereotypical axonal
and dendritic morphology and allowed for an analytic and numerical
analysis, revealing the strong impact of anisotropy on structural
aspects in networks. Certainly such an oversimplified model brings
caveats; as a consequence of the chosen axon shapes, for example,
directions of axonal projections had to be chosen at random, while
statistics from the cortex imply alignment. tissue geometry . However,
focusing the analysis on comparisons with networks lacking anisotropy
in connectivity, the chosen abstractions serve the of this report and
show in allow for clear identification of its i.

With anisotropy likely only being one amongst many potential
morpho\-logy-induced connectivity principles, the results here make an
important case for the study and inclusion of morphological aspects in
future of networks models. The presented connectivity principle may
provide a first step towards a network archetype better resembling in
the local cortical circuits. As connectivity patterns were found to
strongly affect dynamics \parencite{Pernice2011, Zhao2011},
development in this direction is crucial. Furthermore, underlying
principles may reveal unexpected, not yet reported connectivity
statistics. Here we found that the distribution of the number of
common inputs for a random neuron pair shows a drastically higher
variance in networks with a high degree of anisotropy, potentially
enabling an increased functional specificity in the networks. Finally,
connectivity for functional synaptic plasticity providing a likely
important framework. The results presented here thus may help shift 

The results here contribute , shig

Plastic

Thus, important , and shifting a one-sided discussion to
an differentiated in aspects.



% The findings call for more in which generic
% morphology might drastically influence connectivity. Such pre-existing
% restraints and may be essential in the in shaping plastic learning
% processes, Here, we provided an novel





% In this study
% find we found that anisotropy affect a common neighbor rule
% itself. Therefore. promoting to a candidate.


% Pointing out the The results are to be interpreted critically
% however, as an analytical consideration reveals 

% Following an analytical consideration of motif
% distributions in distance-dependent networks, we point out an
% important aspect in the interpretation of such distributions, finding
% an unexpected dependency


% Here, we find that not only edge count overrepresentation are by
% anisotropy, but we further find a distinct common neighbor rule in the
% , . 

 
% In their study \textcite{Perin2011} were able to relate increased
% edge-counts in neuron clusters to a \textit{common neighbor rule} - a
% specific relationship between common in- or outputs in a neuron pair
% and the probability for the pair to be connected - identifying an
% underlying principle for the occurrence of non-random connectivity in
% local circuits. Perin et al.\ suggest that (fire together wire
% together). In this structural analysis we found non-random
% connectivity resembling the findings of Song et al./ and , purely from
% a g. Identifying a common neighbor rule as the inherent, that
% non-random connectivity statistics might arrise from morphological
% restrictions rather than plastic processes.




% Taking distance-dependent networks as a reference, Perin et al.\
% were able to show network connectivity features in cortical circuits
% going beyond distance-dependency. A network model hoping to reflect
% such features, clearly is required to implement connectivity going
% beyond distance-dependent connection probabilities. With an


% Thus, anisotropy in connectivity seems . Initially reporting ,i
%  Perin
% et al.\ identify connectivity features beyond distance . Only a
% network model implementing connectivity rules going beyond
% distance-dependency can possible repo. 

% Introducing such a concept motivated
% from neuronal morphology, we here anisotropy in connectivity
% as an underlying connectivity principle 


% We find that anisotropy in connectivity induces increased high edge
% counts in neuron clusters, matching overrepresentations identified in
% local cortical circuits. 

% Comparing however to distance-dependent as opposed to rewired
% networks, we find that overrepresentation is stronger
% (\autoref{fig:perin_rew_dist}).







%%% Local Variables: 
%%% mode: latex
%%% TeX-master: "../dplths_document"
%%% End: 