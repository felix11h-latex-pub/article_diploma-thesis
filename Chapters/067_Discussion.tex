

\section{Discussion}\label{sec:discussion}

The structural analysis of anisotropic networks revealed connectivity
characteristically different from comparable distance-dependent or
even rewired networks. Such differences however only occur when
considering patterns in connectivity of three or more neurons, pair
connections and standard network measures remain unaffected.

Accordingly, vertex degree distributions in anisotropic networks were
found to have increased variance and skew, but as rewired and
distance-dependent networks also show this feature, this is only
indirectly caused by network anisotropy. Similarly, while
an\-iso\-tro\-pic networks display a small-world property, short path
lengths and high clustering coefficients are not due to anisotropy,
but stem from the network's distance-dependency. In fact we find that
eliminating anisotropy in the network can lead to even shorter path
lengths and higher clustering, strengthening the small-worldness of
the network. Finally, we were able to present conclusive evidence that
anisotropy is not able to reproduce overrepresentation of reciprocally
connected neuron pairs, reported repeatedly in cortical networks.

In terms of higher order connectivity however, anisotropy yields
highly interesting results. Three-neuron motif occurrences are not
only strongly influenced by anisotropy, but resulting
overrepresentations resemble in many aspects the profiles found by
\textcite{Song2005}. Furthermore, anisotropy in connectivity induces
an increased frequency of high edge counts in neuron clusters,
matching results from experiments in the rat's
cortex \parencite{Perin2011}. In their study, Perin et al.\ were able
to relate such edge count overrepresentations to a \textit{common
  neighbor rule} - a specific relationship between shared in- or
outputs of a neuron pair and the probability of connection in the
pair. Being able to reproduce increased high edge counts by imposing
such relation distance-dependent networks, the common neighbor rule
was identified as an underlying organization
principle. Stereotypically connected clusters arising from this rule
then constitute \enquote{building blocks} to be shaped by experience.

In this work, anisotropy was recruited as a connection principle for
the generation of local geometric networks, reflecting typical
cortical slices. Capturing stereotypical axonal and dendritic
morphology in cortical networks, those anisotropic networks inherently
These networks do not only display characteristic patterns , but a
common neighbor relationship is well affect by anisotropy, suggesting
that . 

The findings provide here provide new insight . Calling for more in which
generic morphology might drastically influence connectivity.

  

% In this study
% find we found that anisotropy affect a common neighbor rule
% itself. Therefore. promoting to a candidate.


% Pointing out the The results are to be interpreted critically
% however, as an analytical consideration reveals 

% Following an analytical consideration of motif
% distributions in distance-dependent networks, we point out an
% important aspect in the interpretation of such distributions, finding
% an unexpected dependency


% Here, we find that not only edge count overrepresentation are by
% anisotropy, but we further find a distinct common neighbor rule in the
% , . 

 
% In their study \textcite{Perin2011} were able to relate increased
% edge-counts in neuron clusters to a \textit{common neighbor rule} - a
% specific relationship between common in- or outputs in a neuron pair
% and the probability for the pair to be connected - identifying an
% underlying principle for the occurrence of non-random connectivity in
% local circuits. Perin et al.\ suggest that (fire together wire
% together). In this structural analysis we found non-random
% connectivity resembling the findings of Song et al./ and , purely from
% a g. Identifying a common neighbor rule as the inherent, that
% non-random connectivity statistics might arrise from morphological
% restrictions rather than plastic processes.




% Taking distance-dependent networks as a reference, Perin et al.\
% were able to show network connectivity features in cortical circuits
% going beyond distance-dependency. A network model hoping to reflect
% such features, clearly is required to implement connectivity going
% beyond distance-dependent connection probabilities. With an


% Thus, anisotropy in connectivity seems . Initially reporting ,i
%  Perin
% et al.\ identify connectivity features beyond distance . Only a
% network model implementing connectivity rules going beyond
% distance-dependency can possible repo. 

% Introducing such a concept motivated
% from neuronal morphology, we here anisotropy in connectivity
% as an underlying connectivity principle 


% We find that anisotropy in connectivity induces increased high edge
% counts in neuron clusters, matching overrepresentations identified in
% local cortical circuits. 

% Comparing however to distance-dependent as opposed to rewired
% networks, we find that overrepresentation is stronger
% (\autoref{fig:perin_rew_dist}).







%%% Local Variables: 
%%% mode: latex
%%% TeX-master: "../dplths_document"
%%% End: 