
% ######################################################################### %
% ------------------------------------------------------------------------- %
%                         Summary and Discussion
% ------------------------------------------------------------------------- %
% ######################################################################### %

\section{Summary and Discussion}

This chapter introduced and discussed the following network types:
\begin{enumerate} \index{network types}
  \itemsep-11pt
  \item random networks
  \item distance-dependent networks
  \item rewired anisotropic networks
  \item anisotropic networks
\end{enumerate}
While random and distance-dependent networks were formally as graphs
defined in Sections~\ref{sec:random_graphs} and
\ref{sec:geometric_graphs}, this chapter introduced the concept of
anisotropy. Built upon it, anisotropic graphs were defined as a
geometric graph with a specific connectivity rule. Identifying a
parameter set reflecting local cortical networks, samples of
anisotropic networks and of the other graph types were created. For
this we extracted the distance-dependent connectivity profile in
anisotropic networks and introduced rewiring as a method to eliminate
anisotropy while keeping other connectivity parameters unaltered.
Finally, by introducing the a measure for network anisotropy, we are
able to relate the concepts of distance-dependency and anisotropy,
finding that rewiring does indeed significantly reduce anisotropy
while keeping the distance-dependent connectivity intact.

The network types discussed span a spectrum of completely random
connectivity to networks with specific connectivity rules. In how far
such a specific rule is able to produce non-random connectivity
reflecting findings in local cortical circuits is the main task of
Chapter~\ref{ch:structural_aspects}. For this analysis we will heavily
recruit all of the network models above as well as their computational
implementation. While rewired networks are most closely related to
anisotropic networks and provide a reference for features directly
caused by anisotropy, comparison with distance-dependent and random
networks reveals indirectly affected features and provides insight to
structural features present in anisotropic networks as opposed the
standard network types.


% For this we will recruit all of the networks mentioned above,
% providing a reference to test features against. 

 
% Each of the
% network types were implemented computationally and sample graphs with
% matching parameter set were created. These sample graphs will serve as
% a reference for a structural analysis, spanning the spectrum from
% completely random connectivity to specific connectivity.

% Where rewired networks provide the minimal , distance-dependent
% provide insight into comparison with a more standard model, in many
% cases comparison of structural will give important insight.


% the claim is that distance-dependent and fully rewired are more or
% less equivalent (hints of non-equivalence found in ...) and so




%%% Local Variables: 
%%% mode: latex
%%% TeX-master: "../dplths_document"
%%% End: 
