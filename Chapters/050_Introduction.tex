
\section{Overview}\label{sec:intro_model}

This chapter introduces the central object of this study, the
\textit{anisotropic network model}. Reviewing connectivity in local
cortical circuits in Section~\ref{sec:biol_anisotropy}, we identify
anisotropy in connectivity in layer V pyramidal cells, inferred by
their specific neuronal anatomy. Reducing the complex neuron
morphology to characteristic axonal and dendritic profiles, we
introduce the anisotropic network model in
Section~\ref{sec:anisotropic_network_model}.  While a graph theoretic
definition allows for analytical considerations, a numerical
implementation enables us not only to support the analytical
observations but also gives access to results that go beyond. To fully
harness this implementation we argue for the choice of a specific
parameter set in Section~\ref{sec:numerical_implementation}, allowing
us to generate a set of \enquote{sample graphs} to which we will refer
through this study.

Using the analytical and numerical approach in conjunction, in our
analysis of anisotropic networks we are most interested in identifying
structural aspects that are due to the network's anisotropy in
connectivity and do not occur in similar, isotropic networks. To be
able to make this distinction, in
Section~\ref{sec:distance_connectivity} we extract the
distance-dependent connectivity of anisotropic networks, as it is
imposed by the specific geometric relations present in the network.
We then go on to introduce \enquote{rewiring} in
Section~\ref{sec:rewiring}, a method that allows us to manipulate the
anisotropic networks to eventually display isotropy in
connectivity. Finally then, Section~\ref{sec:anisotropy_measure} ties
together the previous concepts by providing a measure for anisotropy
and showing how rewiring is, in fact, providing the transition from
anisotropic networks to networks with isotropy in connectivity.





%%% Local Variables: 
%%% mode: latex
%%% TeX-master: "../dplths_document"
%%% End: 
