
\section{Overview}\label{sec:intro_model}

This chapter introduces the central object of this study, the
\textit{anisotropic network model}. Reviewing connectivity in local
cortical circuits in Section~\ref{sec:biol_anisotropy}, we identify an
anisotropy in connectivity in layer V pyramidal cells, inferred by
their specific neuronal anatomy. Reducing the complex neuron
morphology to characteristic axonal and dendritic profiles, we
introduce the anisotropic network model in
Section~\ref{sec:anisotropic_network_model}.  While a graph theoretic
definition allows for analytical considerations, a numerical
implementation enables us not only to support the analytical
observations but also yields results that go beyond the analytical
results presented here. To fully harness this implementation we argue
for the choice of a specific parameter set in
Section~\ref{sec:numerical_implementation}, allowing us to generate a
set of \enquote{sample graphs} to which we will refer through this
study.

Using the analytical and numerical approach in conjunction, we are
most interested in identifying structural aspects in the anisotropic
network model that are due to the network's anisotropy in connectivity
and do not occur in similar, isotropic networks. To be able to make
this distinction, in Section~\ref{sec:distance_connectivity} we
extract the distance-dependent connectivity of anisotropic networks,
as it is imposed by the specific geometric relations present in the
network.  We then go on to introduce \enquote{rewiring} in
Section~\ref{sec:rewiring}, a method that allows us to manipulate the
anisotropic networks to eventually display isotropy in connectivity,
yielding us a method for the generation of geometric graphs,
. Finally then, Section~\ref{sec:anisotropy_measure} ties together the
previous by providing a measure for anisotropy and showing how
rewiring is imposing isotropy upon the networks. 





%%% Local Variables: 
%%% mode: latex
%%% TeX-master: "../dplths_document"
%%% End: 
