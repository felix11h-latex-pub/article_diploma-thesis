% ######################################################################### %
% ------------------------------------------------------------------------- %
%                   Anisotropic geometric network model
% ------------------------------------------------------------------------- %
% ######################################################################### %

\newpage
\section{Anisotropic geometric network model}
\label{sec:anisotropic_network_model}

Taking up the concept of anisotropy in neural connectivity introduced
in the last section, we propose here, as basis for this study, a
simple geometric network model featuring anisotropic
connectivity. Constructing such a model, we're challenged with
resembling the anisotropic aspects outlined last section as closely as
possible, while at the same time making the model as simple and as
abstract as possible, to allow for an abstract and analytical study of
such anisotropic networks.

With this in mind, we propose the following model: On a square surface
of side length $s$, a number of $N$ point neurons are randomly,
uniformly distributed.  Connected neighbors are then calculated for
each neuron separately and independently, by determining the randomly,
uniformly distributed direction of the neuron's single axon. In this
direction the axon traverses over the surface describing a straight
path, terminating only when an edge of the surface is
reached. Directed contacts are made with every neuron that is within a
width $w(x)$ of the axon's trajectory, where in general $w$ depends on
the axon length $x$ at this point
(\autoref{fig:anisotropic_network_model}).

\begin{figure}[!htbp]
  \centering 
  \makebox{%
    \begin{overpic}[width=0.325\textwidth,frame]{%
        img/network_model/model_dendrite_w.pdf}
      \put(3,91){\small\textbf{A}}
    \end{overpic}
    \hfill
    \begin{overpic}[width=0.325\textwidth,frame]{%
        img/network_model/model_axon_w.pdf}
      \put(3,91){%
        \fboxsep=0pt\colorbox{white}{\small\textbf{B}}
        }
    \end{overpic}
    \hfill
    \begin{overpic}[width=0.325\textwidth,frame]{%
        img/network_model/connectivity.pdf}
      \put(3,91){\small\textbf{C}}
    \end{overpic}
    }%
    \caption{%
      \textbf{Anisotropic geometric network model and interpretations
        of width parameter $\boldsymbol w$} Illustrating the process
      of finding connections for one neuron (large triangle, black),
      the axon describes a linear trajectory in an arbitrary direction
      and until terminating on the surface's edge. Target neurons
      (red) are encountered along the path within a (here constant)
      distance $w$, which is in \textbf{A)} interpreted as a dendritic
      radius or, equivalently, in \textbf{B)} as a \enquote{bandwith}
      of the axon. Connections to the encountered targets are then
      established as projections in \textbf{C)}, consistent with the
      directed nature of synapses in biological networks
      (cf. Chapter~\ref{ch:Biology}).}
  \label{fig:anisotropic_network_model}
\end{figure}

The implementation of arbitrary axonal orientation is crucial to the
model. Although cortical axons are described as consistently
%------------------------------------------------
\marginpar{random axonal orientation yields relevant connectivity} 
%------------------------------------------------ 
projecting downwards \parencite[%
cf. Section~\ref{sec:biol_anisotropy}]{Braitenberg_Cortex}, combining
exclusively vertically aligned axons with the simplified axonal and
dendritic morphological profiles would result in a \enquote{vertically
  staggered connectivity} - neurons could then only project to targets
located below them.  It is in fact not a vertical alignment of axon
orientation, but the anisotropy in neural connectivity - the
observation of neuronal targets aligning with the axonal projection -
that we try to capture and analyze in this model. 

We will refer to the model as the \textit{anisotropic geometric
  network model}. Trying to provide a simple, abstract model isolating
anisotropy in connectivity, in most of this study the width $w(x)$ is
assumed to be constant, $w(x) =w$, a notable exception being the exploration in
??. In the graph theoretic context the anisotropic network model is a
random graph model, in which a realization of the random
process results in a geometric directed graph with a special mode of
connectivity. We can formally define such realization as:

\begin{definition}[Anisotropic geometric graph]
  \label{def:anisotropic_geometric_graph} 
  \index{anisotropic geometric!graph} %
  Let $n \in \mathbb{N}$ and $w \in (0,\infty)$. An
  \textit{anisotropic geometric graph} $G_{n,w}$ then consists of a
  tuple $(G,\Phi,a)$, of a simple directed graph $G$ with $|V(G)|=n$
  vertices and the maps $\Phi:V(G)\to[0,1]^2$ and $a:V(G)\to[0,2\pi)$,
  such that for every vertex pair $v,v' \in V(G)$ and edge $e\in E(G)$
  with $s(e)=v$ and $t(e)=v'$ exists if and only if the inequalities
  \[
    R_{-a(v)}\left(\Phi(v')-\Phi(v)\right)\hat{e}_x \geq 0 
      \quad \mathrm{and} \quad
    \abs{R_{-a(v)}\left(\Phi(v')-\Phi(v)\right)\hat{e}_y} 
      \leq \frac{w}{2}
  \]
  hold. Here $R_{\varphi}$ is the rotation matrix of angle $\varphi$
  in the Cartesian plane and $\hat{e}_x, \hat{e}_y$ are the standard
  unit vectors. % ?? Alpha or A??
  % for $y=A_{\alpha(v)}(v'-v)$ the identities $y\hat{e}_x > 0$
  % and $\abs{y \hat{e}_y} \le \frac{w}{2}$ hold.
\end{definition}

The anisotropic random graph model then is then giving the probability
distribution over the set of anisotropic random graphs by describing a
random process generating such graph.

\begin{definition}[Anisotropic random graph model]
  \index{anisotropic geometric!random graph model} 
  Let $n \in \mathbb{N}$ and $w > 0$. The \textit{anisotropic random
    graph model} $G(n,w)$ is a probability space over the set of
  anisotropic geometric graphs with a probability distribution induced
  by the following process: Let $G$ be an empty graph with $n$
  vertices. Assign randomly and uniformly to every vertex $v \in V(G)$
  a position $\Phi(v) \in [0,1]^2$ and axonal orientation $0\leq a(v)
  < 2\pi$. Then add edges such that $(G,\Phi,a)$ is an anisotropic
  geometric graph $G_{n,w}$.
\end{definition}


As with every geometric graph model introduced, we restrict the
surface to be the unit square. This does not limit the model, as only
the relative width of the axon band in regard to the surface's side
length is determining connectivity statistics -
%------------------------------------------------
\marginpar{anisotropic model is scale-free} %
%------------------------------------------------ 
the expected number of connections is easily obtained by the quotient
of the area covered by the axon and the surface area, making
connectivity statistics in the anisotropic random graph model
scale-free.

The following maybe interpreted as a study of anisotropic geometric
graphs in the light of a neuroscientific context. To enable such an
analysis, a few more concepts are needed. The introduction of
those concepts composes the rest of the chapter. A first important
step is the numerical implementation of the anisotropic network model.



%%% Local Variables: 
%%% mode: latex
%%% TeX-master: "../dplths_document"
%%% End: 