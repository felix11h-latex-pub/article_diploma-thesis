
% ######################################################################### %
% ------------------------------------------------------------------------- %
%                        Small World Properties
% ------------------------------------------------------------------------- %
% ######################################################################### %

\newpage
\section{Small World Properties}\label{sec:small_world}

Small-world networks, as described in
Section~\ref{sec:small_world_intro}, are characterized by small a
average path length and comparably high clustering coefficient. In the
of brain networks, small-world

\parencite{Watts1998}

Here we are interested in exploring the question if network anisotropy
has effect on the small-worldness of geometric networks. Using
distance-dependent networks as a reference, we find that eliminating
anisotropy through rewiring does affect the small-world properties to
some degree; with rising isotropy in the network, the characteristic
path length declines, while the network clustering coefficient
increases resulting together in rewired networks to display a higher
degree of small-worldness (\autoref{fig:small_world_props}).

\begin{figure}[htp]
  \centering
  \makebox{%
    \hspace{-0.45cm}
    \begin{overpic}[height=0.225\textheight]{%
        plots/064f9b10_apl.pdf}
      %\put(89.8,56.5){\small\textbf{A}}
      \put(14.3,78.9){\small\textbf{A}}
    \end{overpic}
    \hspace{-0.15cm}
    \begin{overpic}[height=0.225\textheight]{%
        plots/064f9b10_lcl_all.pdf}      
       \put(16.2,78.9){\small\textbf{B}}
      %\put(12,5){\small\textbf{B}}
    \end{overpic}
  }%
  \caption{\textbf{Higher degree of small-worldness in
      isotropic than in anisotropic networks} Generating
    anisotropic networks with different axon widths $w$ and extracting
    probability $p$ of dcirected connectin between two random nodes,
    demonstrates the dependency of $p$ on the width parameter $w$.
    \textbf{A)} At an axon width of over $w=100$, exceeding the
    square's side length, the connection probability saturates at
    $p=0.5$, as axon bands are essentially \enquote{cutting} the
    square in a connected and unconnected half
    (\smtcite{c5b64f3e}). \textbf{B)} For small $w$ the connection
    probability is a linear function of $w$, allowing the width $\nicefrac{w_S}{2}$
    at which $p(w_S)=11.6$ to be determined by a linear fit as
    $\nicefrac{w_S}{2} =
    12.6$ (\smtcite{064f9b10}).} %?? fix width issue!!
  \label{fig:small_world_props}
\end{figure}  


In ER networks avg and cc are ...


Comparing path length we find ... 

The analysis . Here we present the influence on anisotropy of

Charactersitic path length



Average path length and so forth

Sporns papers   newest Butz

While the transitivity ratio of a network, its correlation with the
network's anisotropy degree gives hint of deeper structural
relationships found in the subsequent sections.



%%% Local Variables: 
%%% mode: latex
%%% TeX-master: "../dplths_document"
%%% End: 
