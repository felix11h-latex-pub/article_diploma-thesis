
% ######################################################################### %
% ------------------------------------------------------------------------- %
%                        Small World Properties
% ------------------------------------------------------------------------- %
% ######################################################################### %

\newpage
\section{Small World Properties}\label{sec:small_world}

Small-world networks, as described in
Section~\ref{sec:network_measures}, are characterized by a small
average path length and comparably high clustering coefficient. In
brain networks, small-world properties are frequently discussed as the
stemming from few long-rage projections. While most often
reported on the macroscale \parencite{SpornsZwi2004}, small-worldness
is also found in local cortical networks \parencite{Perin2011}.

Here we are interested in exploring the question whether network
an\-iso\-tropy has effect on the small-worldness of geometric
networks. While anisotropic networks display a relatively high
clustering coefficient, $c = \pm $ at network size $N=1000$ (compared
with $p = 0.116$ in random networks) and a comparable path length
$l_{\mathrm{aniso}} = $ and $l_{\mathrm{random}} = 1.8820 \pm 0.0001$
% label:251412db
) in comparison with random graphs, we find that 

However, using distance-dependent networks as a reference, we find
that successively eliminating anisotropy through rewiring contributes
positively to the small-world property; with rising isotropy in the
network, the characteristic path length declines, while the network
clustering coefficient increases resulting together in rewired
networks to display a higher degree of small-worldness
(\autoref{fig:small_world_props}).


\begin{figure}[htp]
  \centering
  \makebox{%
    \hspace{-0.45cm}
    \begin{overpic}[height=0.225\textheight]{%
        plots/064f9b10_apl.pdf}
      %\put(89.8,56.5){\small\textbf{A}}
      \put(14.3,78.9){\small\textbf{A}}
    \end{overpic}
    \hspace{-0.15cm}
    \begin{overpic}[height=0.225\textheight]{%
        plots/064f9b10_lcl_all.pdf}      
       \put(16.2,78.9){\small\textbf{B}}
      %\put(12,5){\small\textbf{B}}
    \end{overpic}
  }%
  \caption{\textbf{Anisotropy does not contribute to small-worldness}
    In increasingly rewired networks, trends show a decreasing average
    path length and rising clustering coefficient and thus possibly a
    higher degree of small-worldness in the rewired, isotropically
    connected networks.  \textbf{A)} Average path lengths for network
    sizes $N=250$, $500$ and $1000$, where vertex pairs with no
    existing are discarded. Individual value pairs are obtained by
    averaging over a trial size of $20$, $15$ and $5$ respectively;
    errorbars are SEM. \textbf{B)} Network configuration as in A),
    additionally showing clustering coefficients for
    distance-dependent networks.
    (\smtcite{064f9b10})} %?? fix width issue!!
  \label{fig:small_world_props}
\end{figure}  


In distance dependent networks the average path length is in general
smaller (\autoref{suppfig:small_world}), matching those of a random
network.

The fact that rewired anisotropic networks show an overall higher
clustering coefficient as purely distance-dependent networks is not
fully explained, but presumably relates to the difference in the
out-degree distribution (\autoref{fig:out_degree_rewiring}),


%Conclusions:

%Ansiotropy does not significantly influence the average path
%length. While high clustering coefficient is observed,


%%% Local Variables: 
%%% mode: latex
%%% TeX-master: "../dplths_document"
%%% End: 
