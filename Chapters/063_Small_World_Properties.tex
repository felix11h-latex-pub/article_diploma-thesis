
% ######################################################################### %
% ------------------------------------------------------------------------- %
%                        Small World Properties
% ------------------------------------------------------------------------- %
% ######################################################################### %

\newpage
\section{Small World Properties}\label{sec:small_world}

Small-world networks, as described in
Section~\ref{sec:network_measures}, are characterized by a small average path length and comparably high clustering coefficient. In the
of brain networks, small-world

\parencite{Watts1998}

Here we are interested in exploring the question if network anisotropy
has effect on the small-worldness of geometric networks. Using
distance-dependent networks as a reference, we find that successively eliminating
anisotropy through rewiring does affect the small-world properties to
some degree; with rising isotropy in the network, the characteristic
path length declines, while the network clustering coefficient
increases resulting together in rewired networks to display a higher
degree of small-worldness (\autoref{fig:small_world_props}).


\begin{figure}[htp]
  \centering
  \makebox{%
    \hspace{-0.45cm}
    \begin{overpic}[height=0.225\textheight]{%
        plots/064f9b10_apl.pdf}
      %\put(89.8,56.5){\small\textbf{A}}
      \put(14.3,78.9){\small\textbf{A}}
    \end{overpic}
    \hspace{-0.15cm}
    \begin{overpic}[height=0.225\textheight]{%
        plots/064f9b10_lcl_all.pdf}      
       \put(16.2,78.9){\small\textbf{B}}
      %\put(12,5){\small\textbf{B}}
    \end{overpic}
  }%
  \caption{\textbf{Anisotropy does not contribute to small-worldness}
    In increasingly rewired networks, trends show a decreasing average
    path length and rising clustering coefficient and thus possibly a
    higher degree of small-worldness in the rewired, isotropically
    connected networks.  \textbf{A)} Average path lengths for network
    sizes $N=250$, $500$ and $1000$, where vertex pairs with no
    existing are discarded. Individual value pairs are obtained by
    averaging over a trial size of $20$, $15$ and $5$ respectively;
    errorbars are SEM. \textbf{B)} Network configuration as in A),
    additionally showing clustering coefficients for
    distance-dependent networks.
    (\smtcite{064f9b10})} %?? fix width issue!!
  \label{fig:small_world_props}
\end{figure}  


In distance dependent networks the average path length is in general
smaller (\autoref{suppfig:small_world}), matching those of a random network.


In ER networks avg and cc are ...


Comparing path length we find ... 

The analysis . Here we present the influence on anisotropy of

Charactersitic path length



Average path length and so forth

Sporns papers   newest Butz

While the transitivity ratio of a network, its correlation with the
network's anisotropy degree gives hint of deeper structural
relationships found in the subsequent sections.



%%% Local Variables: 
%%% mode: latex
%%% TeX-master: "../dplths_document"
%%% End: 
