
% ######################################################################### %
% ------------------------------------------------------------------------- %
%                        Small World Properties
% ------------------------------------------------------------------------- %
% ######################################################################### %

\section{Small World Properties}\label{sec:small_world}

Small-world networks, as described in
Section~\ref{sec:network_measures}, are characterized by a small
average path length and comparably high clustering coefficient. In
brain networks, combining advantages of sparse connectivity with
mostly local and only few long-range projections, small-world
properties are frequently discussed as a way of achieving high
efficiency in the parallel processing of
information \parencite[cf.][]{Achard2007}. While most often reported
on the macroscale \parencite{SpornsZwi2004, Bassett2006},
small-worldness is also found in local cortical
networks \parencite[SI]{Perin2011}.

Here we are interested in exploring the question whether
an\-iso\-tropy in connectivity influences the small-worldness of
geometric networks. First we find that at a network size $N=1000$,
% ----------------------------------------
\marginpar{in random networks from independence clustering = $p$}
% ----------------------------------------
anisotropic networks display a relatively high clustering coefficient,
$c = 0.1581 \pm 0.0008$ compared with $p = 0.116$ in random networks,
and a comparable path length, $l_{\mathrm{aniso}} = 1.937 \pm 0.002$
and $l_{\mathrm{random}} = 1.8820 \pm 0.0001$, % label:251412db
ascertaining a small-world property in the anisotropic network model.

\begin{figure}[H]
  \centering
  \makebox{%
    \hspace{-0.45cm}
    \begin{overpic}[height=0.225\textheight]{%
        plots/064f9b10_apl.pdf}
      %\put(89.8,56.5){\small\textbf{A}}
      \put(14.3,78.9){\small\textbf{A}}
    \end{overpic}
    \hspace{-0.15cm}
    \begin{overpic}[height=0.225\textheight]{%
        plots/064f9b10_lcl_all.pdf}      
       \put(16.2,78.9){\small\textbf{B}}
      %\put(12,5){\small\textbf{B}}
    \end{overpic}
  }%
  \captionsetup{skip=8pt}
  \caption{\textbf{Anisotropy does not contribute to small-worldness}
    In increasingly rewired networks, trends show a decreasing average
    path length and rising clustering coefficient and thus possibly a
    higher degree of small-worldness in the rewired, isotropically
    connected networks.  \textbf{A)} Average path lengths for network
    sizes $N=250$, $500$ and $1000$, where vertex pairs with no
    existing are discarded. Individual value pairs are obtained by
    averaging over a trial size of $20$, $15$ and $5$ respectively;
    errorbars are SEM. \textbf{B)} Network configuration as in A),
    additionally showing clustering coefficients for
    distance-dependent networks.
    (\smtcite{064f9b10})} %?? fix width issue!!
  \label{fig:small_world_props}
\end{figure} 

However, is this degree of small-worldness inferred by anisotropy in
connectivity? Using distance-dependent networks as a reference, we
find that successively eliminating anisotropy through rewiring
contributes positively to the small-world property; with rising
isotropy in the network, the characteristic path length declines in
small networks and remains unchanged in larger networks, while the
clustering coefficient increases regardless of network size, resulting
together in rewired networks to display a higher degree of
small-worldness (\autoref{fig:small_world_props}).


In distance-dependent networks the average path length is generally
smaller than in (rewired) anisotropic networks
(\autoref{suppfig:small_world}), matching those of a random network as
reported above. At the same time also the clustering coefficient is
smaller than in anisotropic networks
(\autoref{fig:small_world_props}), resulting overall in a comparable
degree of small-worldness in distance-dependent networks
\footnote{Differences in the absolute values of both path length and
  clustering coefficient presumably relates to difference in the
  out-degree distribution (\autoref{fig:out_degree_rewiring})} and
leading to the conclusion that the observed small-worldness in the
anisotropic networks is due to the imposed distant-dependent
connectivity rather than the anisotropy in connectivity.




%Conclusions:

%Ansiotropy does not significantly influence the average path
%length. While high clustering coefficient is observed,


%%% Local Variables: 
%%% mode: latex
%%% TeX-master: "../dplths_document"
%%% End: 
