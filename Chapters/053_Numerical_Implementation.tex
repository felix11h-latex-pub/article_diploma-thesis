

% ######################################################################### %
% ------------------------------------------------------------------------- %
%                         Numerical Implementation
% ------------------------------------------------------------------------- %
% ######################################################################### %


\section{Numerical Implementation}\label{numerical_implementation}

Numerical implementation of the anisotropic random graph model was
achieved in Python\footnote{Python Software Foundation. Python
  Language Reference, version 2.7. Available at
  \url{http://www.python.org}}. Relying on NumPy as part of the
scientific Python library SciPy\footnote{Eric Jones, Travis Oliphant,
  Pearu Peterson and others. NumPy version 1.6.1. Available at \url{http://www.scipy.org}}
for the more complex mathematical computations, the implementation
also uses graph-tool\footnote{Tiago
  P. Peixoto. Efficient network analysis. Version 2.2.18. Available at
  \url{http://graph-tool.skewed.de/}}, to ensure convenient and
efficient handling of the created networks.

The algorithm for the generation of anisotropic networks closely
resembles Definition~\ref{def:anisotropic_geometric_graph}. After
randomly distributing $N$ neurons on the square of side-length $s$,
for every neuron a random axon horientation $a \in [0,2\pi)$ is chosen
and an affine transformation, such that the current neuron is located
at the origin and its axon projection aligns with the positive x-axis,
secures a straightforward implementation of connectivity, using the
the inequalities in Definition~\ref{def:anisotropic_geometric_graph}
as a rule for establishing connections.

To harness the numerical implemenation to generate networks, a set of
parameters needs to be chosen. The network size $N$ strongly
influences the needed computational efforts in calculations based on
the generated graphs and has thus been set to $N = 1000$. 
%------------------------------------------------
\marginpar{parameter set chosen to resemble cortical circuits}
% ------------------------------------------------
Choosing the surface side-length arbitrarily as $s=100$, the axon
width $w$ determines connectivity in the network, the relation between
width $w$ and overall connection probability $p$ being shown in
\autoref{fig:determine_axon_width}.  In their analysis of connectivity
of thick-tufted layer V pyramidal cells in neonatal rats (day 14),
\textcite{Song2005} report an overall connection probability of
$p=0.116$, consistent with prior reports of a cortical connection
probability of $p \approx 0.1$. Choosing $w$ to be constant, we
determine the axon width such that overall connectivity matches the
value report by Song et al. and obtain $w/2 = 12.6$
(\autoref{fig:determine_axon_width}).

\begin{figure}[htp]
  \centering
  \makebox{%
    \begin{overpic}[width=0.5\textwidth]{%
        plots/c5b64f3e.pdf}
      \put(22.5,57.5){\small\textbf{A}}
      %\put(12,5){\small\textbf{A}}
    \end{overpic}
    \hfill
    \begin{overpic}[width=0.5\textwidth]{%
        plots/585a946f.pdf}
      \put(24.5,57.5){\small\textbf{B}}
      %\put(12,5){\small\textbf{B}}
    \end{overpic}
  }%
  \vspace{-0.15cm}
  \caption{\textbf{Axon width dependent connection probability
      determines parameter for numerical analysis} Generating
    anisotropic networks with different axon widths $w$ and extracting
    probability $p$ of directed connection between two random nodes,
    demonstrates the dependency of $p$ on the width parameter $w$.
    \textbf{A)} At an axon width of over $w=100$, exceeding the
    square's side length, the connection probability saturates at
    $p=0.5$, as axon bands are essentially \enquote{cutting} the
    square in a connected and unconnected half
    (\smtcite{c5b64f3e}). \textbf{B)} For small $w$ the connection
    probability is a linear function of $w$, allowing the width $\nicefrac{w_S}{2}$
    at which $p(w_S)=11.6$ to be determined by a linear fit as
    $\nicefrac{w_S}{2} =
    12.6$ (\smtcite{585a946f}).} %?? fix width issue!!
  \label{fig:determine_axon_width}
\end{figure}



\label{sample_graphs}Having determined a suitable set of parameters as
$N=1000$, $s=100$ and $w=25.2$, we generate 25 graphs with this parameter
set (label: \smtcite{N1000w\char`_ax126-flat\char`_graph0-24}).
%------------------------------------------------
\marginpar{sample graphs as reference for structural analysis}
%------------------------------------------------ 
This set of sample graphs will serve as a reference for the following
structural analysis. Extending the set by the (partially) rewired
sample graphs (see Section~\ref{sec:rewiring}) and by
purely distance-dependent graphs best resembling the anisotropic
networks (see Section~\ref{distance_dependent_sample}) we obtain a resourceful
reference for the analysis of structural features of anisotropic
geometric graphs, that we will frequently employ to obtain
quantitative and qualitative results.











% \cite{Thomson2002} Axons are not looking for their targets but
% dendrites might, further evidence to focus on axon geometry.


%%% Local Variables: 
%%% mode: latex
%%% TeX-master: "../dplths_document"
%%% End: 
