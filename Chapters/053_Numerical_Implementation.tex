

% ######################################################################### %
% ------------------------------------------------------------------------- %
%                         Numerical Implementation
% ------------------------------------------------------------------------- %
% ######################################################################### %


\section{Numerical Implementation}\label{numerical_implementation}

Numerical implementation of the anisotropic random graph model was
achieved in Python\footnote{Python Software Foundation. Python
  Language Reference, version 2.7. Available at
  \url{http://www.python.org}}. Relying on the NumPy for mathematical
as part of the scientific library \parencite{scipy} %?? Year format?
and graph tool, a graph manipulation \parencite{graph_tool},
convenience as well as potency could be secured.

The algorithm for the generation of anisotropic networks, closely
zresembles the Definition~\ref{def:anisotropic_geometric_graph}. 


To harness the numerical implemenation to generate networks, a set of
parameters needs to be chosen. The network size $N$ strongly
influences
%------------------------------------------------
\marginpar{determine parameter set to generate sample graphs}
%------------------------------------------------ 
the needed computational efforts in calculations based on the
generated graphs and has thus been set to $N = 1000$. Choosing the
surface side-length arbitrarily as $s=100$, the axon width $w$
determines connectivity in the network, the relation between width $w$
and overall connection probability $p$ being shown in
\autoref{fig:determine_axon_width}. In their analysis of connectivity
of thick-tufted layer V pyramidal cells in neonatal rats (day 14),
\textcite{Song2005} report an overall connection probability of
$p=0.116$, consistent with prior reports of a cortical connection
probability of $p \approx 0.1$. Choosing $w$ to be constant, we
determine the axon width such that overall connectivity matches the
value report by Song et al. and obtain $w = 12.6$
(\autoref{fig:determine_axon_width}).

\begin{figure}[!htbp]
  \centering
  \makebox{%
    \begin{overpic}[width=0.5\textwidth]{%
        plots/c5b64f3e.pdf}
      \put(22.5,57.5){\small\textbf{A}}
      %\put(12,5){\small\textbf{A}}
    \end{overpic}
    \hfill
    \begin{overpic}[width=0.5\textwidth]{%
        plots/585a946f.pdf}
      \put(24.5,57.5){\small\textbf{B}}
      %\put(12,5){\small\textbf{B}}
    \end{overpic}
  }%
  \vspace{-0.15cm}
  \caption{\textbf{Axon width dependent connection probability
      determines parameter for numerical analysis} Generating
    anisotropic networks with different axon widths $w$ and extracting
    probability $p$ of directed connection between two random nodes,
    demonstrates the dependency of $p$ on the width parameter $w$.
    \textbf{A)} At an axon width of over $w=100$, exceeding the
    square's side length, the connection probability saturates at
    $p=0.5$, as axon bands are essentially \enquote{cutting} the
    square in a connected and unconnected half
    (\smtcite{c5b64f3e}). \textbf{B)} For small $w$ the connection
    probability is a linear function of $w$, allowing the width $\nicefrac{w_S}{2}$
    at which $p(w_S)=11.6$ to be determined by a linear fit as
    $\nicefrac{w_S}{2} =
    12.6$ (\smtcite{585a946f}).} %?? fix width issue!!
  \label{fig:determine_axon_width}
\end{figure}



Having determined parameters...
With this parameter set we generate a sample of 25 graphs %?? Sumatra:
                                %"N1000w_ax126-flat_graph0-24"








% As shown in ??, only the
% quotient of the side length parameter $e$ and axon width $w$ . As
% such, side length $e$ is arbitrarily set as $e = 100$ and leaves axon
% width for determination.


% First, we determine $w$ to be constant. Although simplistic profiles
% (\autoref{fig:axon_heat}) and makes for characteristic distance
% distribution as we will see later
% (\autoref{fig:distance_theory_compare}), more in line with idea of
% abstractness and simplisticity. For small $w$ then, the overall
% connection probability $p$ can be approximated as
% \[
% p = \frac{L w}{E^2},
% \]
% where $L$ is the average length of the axon until it terminates on a
% surface edge. 

% Having established the connection between , 

% The final parameter is then the axon-profile width $w$. In their
% analysis of connectivity of thick-tufted layer V pyramidal cells in
% neonatal rats (day 14), \textcite{Song2005} report an overall connection
% probability of $p=0.116$, consistent with prior reports of a cortical
% connection probability of $p \approx 0.1$. %?? Which reports?  


% \cite{Thomson2002} Axons are not looking for their targets but
% dendrites might, further evidence to focus on axon geometry.


%%% Local Variables: 
%%% mode: latex
%%% TeX-master: "../dplths_document"
%%% End: 
