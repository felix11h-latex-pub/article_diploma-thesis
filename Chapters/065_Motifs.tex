

\section{Motifs}

In this chapter we analyze the strucarl. The term motif referes
to... . Studies of \textcite{Song2005} and \textcite{Perin2011} show
stuff.

\subsection*{Three-neuron patterns}

Song motifs:


\begin{figure}[H]
  \centering
  \includegraphics[width=0.95\linewidth]{%
    plots/4839ce41_aniso_rand.pdf} 
  \captionsetup{skip=8pt}
  \caption{\textbf{Relative occurrence of three-neuron patterns}
    Extracting the counts of three-node motifs in anisotropic (filled
    bars) and distance-dependent networks (unfilled bars), the
    quotient of the obtained count with the number of occurrences
    expected from the two-neuron connection probabilities in the
    networks (rs =, ,, cf.) shows the over- and underrepresentation of
    specific motifs in the network (red and black errorbars are
    SEM). In anisotropic networks pattern number \enquote{12}, for
    example, appears around five times more often than we would expect
    from the occurrence two-neuron connections. The relative counts
    for anisotropic networks resemble the findings of
    \textcite{Song2005} and differ significantly from the counts in
    distance-dependent networks, implying that anisotropy has a strong
    influence on the relative occurrence of three-neuron
    patterns. (\smtcite{4839ce41}) }
  \label{fig:distance_theory_compare}
\end{figure}



%%% Local Variables: 
%%% mode: latex
%%% TeX-master: "../dplths_document"
%%% End: 
