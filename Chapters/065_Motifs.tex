

\section{Motifs}\label{sec:motifs}

In this chapter we analyze the strucarl. The term motif referes
to... . Studies of \textcite{Song2005} and \textcite{Perin2011} show
stuff. Pernice2011, Sporns , Zhao2011.

% \newpage
% \subsection*{Three-neuron patterns}

% Here we investigate the occurrence of three-neuron patterns in
% an\-iso\-tro\-pic networks. \textcite{Song2005} reported a
% characteristic, highly non-ran\-dom motif distribution of pyramidal
% cells in the rat's visual cortex (layer 5), a result later confirmed
% by \textcite{Perin2011} in their experiment in the rat's somatosensory
% cortex (layer 5). Repeating the experiment \textit{in silico} for the
% different networks subject to this study, we find similar,
% characteristic motif distributions strongly influenced by the
% anisotropy in connectivity.

% There are 13%
% %------------------------------------
% \footnote{%
%   There are 16 simple directed with 3 nodes. Three of those graphs are
%   unconnected \parencite[cf. ][%
%   N. J. A. Sloane. The On-Line Encyclopedia of Integer Sequences,
%   http://oeis.org. Sequence
%   \href{http://oeis.org/A000273}{A000273}]{Davis1953}.%
% } % 
% %-------------------------------------
% three-neuron motifs that represent non-isomorphic, connected simple
% directed graphs. In reference to Song et al.'s result, the patterns
% are labeled 4 to 16, \vspace{-0.2cm}
% \begin{figure}[H]
%   \centering
%   \begin{overpic}[width=0.95\linewidth]{%
%     plots/4839ce41_motifs_cropped.pdf}
%   \put(100,0.2){.} 
%   \end{overpic}
% \end{figure}
% \vspace{-0.8cm} Let $X$ be a random variable that maps three random
% vertices $v_1 \neq v_2 \neq v_3$ in a graph $G$ to the $n \in
% \{4,5,\dots,16\}$ labeling the isomorphism class of their spanned
% subgraph in $G$ as above if the subgraph is connected, and let $X$ map
% to $n=0$ otherwise. A first idea of how to compute the distribution of
% $X$ is by inferring the probabilities of motif occurrence from the
% two-neuron connection probabilities from
% Section~\ref{sec:two_neuron}. In anisotropic networks we found that
% the probabilities of occurrence are \vspace{-0.6cm}
% \begin{align*} 
%   p_u & = 0.791336     &&\text{for unconnected pairs,}     \\
%   p_s & = 0.184151     &&\text{for single connections and} \\
%   p_r & = 0.024513     &&\text{for reciprocal connections.}
% \end{align*}
% From these we may, for example, calculate the probability of
% occurrence for motif 8, 
% \[
%   \mathbf{P}(X=8) = 6\, p_{u} p_{s} p_{r},
% \]
% where the factor 6 is determined by the number of different
% \textit{labeled} graphs belonging to the isomorphism class. The
% distribution of $X$ for the remaining motifs is given by \\
% %
% \smallskip
% %
% \begin{minipage}{\linewidth}
%   \begin{minipage}[c]{0.32\textwidth}
%     \begin{align*}
%       % \mathbf{P}(X=1) &    =   p_u^3  \\
%       % \mathbf{P}(X=2) &    =   6 p_u p_u p_s\\
%       % \mathbf{P}(X=3) &    =   3 p_u p_u p_r\\
%       \mathbf{P}(X=4) &    =   3 p_s^2 p_u\\
%       \mathbf{P}(X=5) &    =   3 p_s^2 p_u\\
%       \mathbf{P}(X=6) &    =   6 p_s^2 p_u\\
%       \mathbf{P}(X=7) &    =   6 p_s p_u p_r
%     \end{align*}
%   \end{minipage}%
%   \begin{minipage}[c]{0.32\textwidth}
%     \begin{align*}
%       \mathbf{P}(X=\,\,\,9) &    =   3 p_r^2 p_u\\
%       \mathbf{P}(X=10) &   =   6 p_s^3   \\
%       \mathbf{P}(X=11) &   =   2 p_s^3    \\
%       \mathbf{P}(X=12) &   =   3 p_s^2 p_r
%     \end{align*}
%   \end{minipage}%
%   \begin{minipage}[c]{0.32\textwidth}
%     \begin{align*}
%       \mathbf{P}(X=13) &   =   6 p_s^2 p_r\\
%       \mathbf{P}(X=14) &   =   3 p_s^2 p_r\\
%       \mathbf{P}(X=15) &   =   6 p_s p_r^2\\
%       \mathbf{P}(X=16) &   =   p_r^3.
%     \end{align*}
%   \end{minipage}  
% \end{minipage}


% Does this distribution accurately reflect the occurrences of
% three-neuron motifs in anisotropic or even distance-dependent
% networks? Here we take the distribution \marginpar{distribution from
%   neuron-pairs as reference} determined from the two-neuron
% probabilities as a reference to analyze occurrences of three-neuron
% motifs in our sets of sample graphs. Counting the occurrences of
% patterns in we find that there are significant over- and
% underrepresentations in anisotropic as well as distance-dependent
% networks, relative to our expectation
% (\autoref{fig:3motif_single}). We find, for example, that in
% anisotropic graphs pattern 12 occurs almost 5 times as often as we
% would have expected from the two-neuron probabilities, whereas
% the counts for pattern 11 only make up less than $30\%$ of the
% occurrences expected.  

% \begin{figure}[H]
%   \centering
%   \includegraphics[width=0.95\linewidth]{%
%     plots/4839ce41_aniso_dist.pdf}
%   \captionsetup{skip=8pt}
%   \caption{\textbf{Relative occurrence of three-neuron patterns}
%     Extracting the counts of three-node motifs in anisotropic (filled
%     bars) and distance-dependent networks (unfilled bars), the
%     quotient of the obtained count with the number of occurrences
%     expected from the two-neuron connection probabilities in the
%     networks (cf. Section~\ref{sec:two_neuron}) shows the over- and
%     underrepresentation of specific motifs in the network (red and
%     black errorbars are SEM). In anisotropic networks pattern 12, for
%     example, appears around five times more often than we would expect
%     from the occurrence two-neuron connections. The relative counts
%     for anisotropic networks resemble the findings of
%     \textcite{Song2005} and differ significantly from the counts in
%     distance-dependent networks, implying that anisotropy has a strong
%     influence on the relative occurrence of three-neuron
%     patterns. (\smtcite{4839ce41}) }
%   \label{fig:3motif_single}
% \end{figure}

% Comparing the relative counts for motifs in anisotropic graphs with
% those in comparable distance-dependent networks, we
% \marginpar{anisotropy strongly affects 3-motif
%   occurrence}identify a strong influence of anisotropy in connectivity
% on three-neuron motif occurrence (\autoref{fig:3motif_single}). In
% their experiments, Song et al.\ and Perin et al.\ find an
% overrepresentation of motifs 4, 10, 12 and 14. In anisotropic networks
% increased counts of motifs 4, 8, 10, 12, 13 and 14 were
% recorded. However, motifs 8 and 13 are overrepresented in
% distance-dependent networks as well, leaving the reported motifs 4,
% 10, 12 and 14 as motifs that are overrepresented due to anisotropy. To
% analyze this effect closer, we also compare three-neuron counts before
% and after rewiring in anisotropic networks
% (\autoref{fig:3motif_full}).


% \begin{figure}[H]
%   \centering
%   \begin{overpic}[width=0.95\linewidth]{%
%       plots/4839ce41_aniso_rew.pdf}
%     \put(2.2,40){\small \textbf{A}} 
%   \end{overpic}
%   \begin{overpic}[width=0.95\linewidth]{%
%       plots/4839ce41_tanfit_rew.pdf}
%     \put(2.2,40){\small \textbf{B}}
%   \end{overpic}
%   \vspace{0.4cm}

%   \makebox{%
%     \hspace{1.208cm}%1.23
%     \begin{overpic}[height=3.6cm]{%
%         plots/4839ce41_motif15.pdf}
%       \put(-10.8,45){\small \textbf{C}}
%       \put(32.3,44.9){\small 15}
%       \put(40,42.9){\includegraphics[width=0.5cm]{img/misc/song_motif_15.pdf}}
%     \end{overpic} 
%     \hspace{0.19cm}
%     \begin{overpic}[height=3.6cm]{%
%         plots/4839ce41_motif16.pdf}
%       \put(54,83){\small 16}
%       \put(68,79){\includegraphics[width=0.5cm]{img/misc/song_motif_16.pdf}}
%     \end{overpic} 
%   }%

%   \captionsetup{skip=8pt}
%   \caption{\textbf{Three-neuron motif occurrence in different network
%       types} \textbf{A)} Comparing counts in anisotropic sample graphs
%     with their rewired counterparts. \textbf{B)} Three-neuron motifs
%     occurrence in tuned anisotropic networks
%     (cf. Section~\ref{sec:tuned_networks}) with their rewired
%     counterparts. For this two-neuron connection probabilities were
%     extracted as in Section~\ref{sec:two_neuron} and motif
%     probabilities were calculated analogously to anisotropic
%     networks. \textbf{C)} Relative counts for the high edge count
%     motifs 15 and 16 for different network types, errorbars
%     SEM. (\smtcite{4839ce41}) }
%   \label{fig:3motif_full}
% \end{figure}


% Considering motif occurrences in anisotropic as well as tuned
% anisotropic networks, we once again confirm the overrepresentation of
% motifs 4, 10, 12 and 14. However, increased counts of pattern 4 are
% observed in the rewired networks as well, leading to the conclusion
% that increased occurrence in this motif is only implicitly affected by
% anisotropy. Motifs 10, 12 and 14 however show significant
% overrepresentation even over their rewired counterparts in anisotropic
% as well as in tuned anisotropic networks.

% The overall motif distribution shows itself stable under changes in
% the distance-dependency with the notable exception of motif 9, that
% shows underrepresentation only in anisotropic but not in tuned
% anisotropic or any distance-dependent network type. Analyzing the
% occurrences of motifs 15 and 16 with a high edge counts
% (\autoref{fig:3motif_full} C) we find that anisotropy has strong
% influence on both motifs, with motif 15 being significantly
% overrepresented in anisotropic networks. Motif 16 shows a highly
% increased occurrence in anisotropic networks, however tuning causes
% the loss of this feature in the network connectivity.

% Summarizing the above observations, we find that \marginpar{results
%   summary} anisotropy in connectivity induces increased occurrence of
% motifs 10, 12, 14 and 15 in the network, reflecting experimental
% results in the rat's cortex.  While over- and underrepresentation
% observed in local cortical circuits can be indirectly linked to
% anisotropy for some motifs (4, 9) it does not accurately reflect
% observed counts for other motifs (8) and shows instability under
% manipulation of distance-dependency in some patterns (9, 16).









\newpage
\subsection*{Edge counts in neuron clusters}

we find resembling Perin et al.'s observation. We calculate 

\begin{figure}[H]
  \centering
  \begin{overpic}[width=0.95\linewidth]{%
      plots/Perin_test_6counts_compare.pdf} 
    \put(3.2,43){\small \textbf{A}}
  \end{overpic}
  \begin{overpic}[width=0.95\linewidth]{%
    plots/7c826e10_test.pdf} 
    \put(3.2,43){\small \textbf{B}}
  \end{overpic}
  \begin{overpic}[width=0.95\linewidth]{%
    plots/7c826e10_test.pdf} 
    \put(3.2,43){\small \textbf{C}}
  \end{overpic}
  \captionsetup{skip=8pt}
  \caption{\textbf{Increased occurrence of high edge counts in neuron
      clusters in anisotropic networks}
    Showing the quotient of the difference 
    Extracting the counts of three-node motifs in anisotropic (filled
    bars) an (\smtcite{4839ce41}) }
  \label{fig:perin6to12}
\end{figure}


\begin{figure}[H]
  \centering
  \begin{overpic}[width=0.95\linewidth]{%
      plots/7c826e10_dist_rew_comp.pdf} 
    \put(3.2,43){\small \textbf{A}}
  \end{overpic}
  \captionsetup{skip=8pt}
  \caption{\textbf{Increased occurrence of high edge counts in neuron
      clusters in anisotropic networks}
    Showing the quotient of the difference 
    Extracting the counts of three-node motifs in anisotropic (filled
    bars) an (\smtcite{7c826e10}) }
  \label{fig:perin_rew_dist}
\end{figure}


In their study, Perin et al.\ follow the observation of increased edge
counts in neuron clusters with a common neighbor rule. Hebb
(\enquote{fire together, wire together}). Here we also investigate our
networks for the existence of a common neighbor relationship. 



% \begin{figure}[H]
%   \centering
%   \renewcommand{\tabcolsep}{0pt}
%   \setlength\extrarowheight{0pt}
%   \begin{tabular}{ll}
%     \begin{overpic}[width=0.5\textwidth]{%
%         /users/hoffmann/research/cn_k_test.pdf}
%       %\put(12,56){\small $\eta = 0$}
%     \end{overpic}
%     &
%     \begin{overpic}[width=0.5\textwidth]{%
%         /users/hoffmann/research/cn_k_test.pdf}
%       %\put(12,56){\small $\eta = 0.25$}
%     \end{overpic}
%     \\
%     \begin{overpic}[width=0.5\textwidth]{%
%         /users/hoffmann/research/cn_k_test.pdf}
%       %\put(12,56){\small $\eta = 0.5$}
%     \end{overpic}
%     &
%     \begin{overpic}[width=0.5\textwidth]{%
%         /users/hoffmann/research/cn_k_test.pdf}
%       % \put(12,56){\small $\eta = 0.75$}
%       %\put(4,-4){\small$0$}\put(78,-4){\small$200$}
%     \end{overpic}
%     \\
%     % \begin{overpic}[width=0.28\textwidth]{%
%     %     plots/77995b6b_in100.pdf}
%     %   \put(12,56){\small $\eta = 1$}
%     %   \put(4,-4){\small$0$}\put(78,-4){\small$200$}
%     % \end{overpic}
%     % & 
%     % \begin{overpic}[width=0.28\textwidth]{%
%     %     plots/77995b6b_indst.pdf}
%     %   \put(12,56){\small distance}
%     %   \put(4,-4){\small$0$}\put(78,-4){\small$200$}
%     % \end{overpic}
%     % \\
%   \end{tabular}
%   \caption{\textbf{In-degxree distrbution not affected by varying
%       degrees of anisotropy} 
%     (\smtcite{77995b6b}). }
%   \label{fig:in_degree_rewiring}
% \end{figure}




%%% Local Variables: 
%%% mode: latex
%%% TeX-master: "../dplths_document"
%%% End: 
