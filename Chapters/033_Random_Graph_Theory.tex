



% ######################################################################### %
% ------------------------------------------------------------------------- %
%                          Random Graph Theory
% ------------------------------------------------------------------------- %
% ######################################################################### %


\section{Random Graph Theory}\label{sec:random_graph_theory}


From this chapter on, as it is common and practical when talking about
random graph models, we move away from the abstract notion
%------------------------------------------------
\marginpar{focus on labeled, simple directed graphs}
%------------------------------------------------ 
of graphs and their equivalence classes and consider \textit{labeled
  graphs}, where the vertex set of a graph with $n$ vertices takes the
form $V = \{1,\ldots,n\}$. Simple directed graphs constitute the
fundamental mathematical object underlying the concepts developed in
this work and if not specified otherwise, all graphs are assumed to be
labeled and simple directed.
%?? Focus on simple directed graphs!

The concept of a random graph was first formally introduced by
\textcite{Erdos1959}. In their $G(n,N)$ model, an undirected graph
with $n$ vertices and $N$ edges is randomly and with equal probability
selected from the set of such possible graphs. In the same year
\textcite{Gilbert1959} independently introduced his $G(n,p)$ model,
realizing edges between vertex pairs with a fixed probability $p$. The
two models are closely related \parencite{Luczak1990} and overlap in
literature, both at times being to referred to as
\textit{Erd\H{o}s-R\'{e}nyi graphs}. Here we focus on the $G(n,p)$
model, as it closer in concept to a computational implementation of a
random graph. Defining it in detail in \ref{def:gilbert_random_graph},
we will refer to the random graph model as the \textit{Gilbert random
  graph model}.

\index{random graph}%
In general, a random graph model is a probability space over a set of
graphs \parencite{Janson_Random-graphs}. Rather than specifying the
sample space and probability
%------------------------------------------------
%\marginpar{definition as probability space or random process}
%------------------------------------------------ 
measure explicitly, random graph models are often defined by a random
process that generates such graphs, leaving probability measure and
sample space implicit \parencite{Bollobas_Random-graphs}. The term
\textit{random graph}, in the graph theoretical context, refers to the
random graph model itself. Especially in the computational context
however, a random graph often refers to a single graph generated by a
random process. Here we try to avoid this ambiguity and strictly refer
to the mathematical object as a random graph model.

Keeping in mind that the term \textit{graph} now refers to labeled,
simple directed graphs if not otherwise specified we define $G^n$ to
be the set of simple directed graphs with $n$ vertices,
\[
G^n := \{G \mid G\,\mathrm{\,graph},\, |V(G)| = n\}.
\]
We first introduce Gilbert's random graph model $G(n,p)$ by explicitly
defining a probability space over $G^n$ and show later how the model
may be realized as a random process. 

\begin{definition}[Gilbert random graph model]
  \label{def:gilbert_random_graph} \index{Gilbert random graph model}
  Let $n\in\mathbb{N}$ and $0\leq p \leq 1$. The \textit{Gilbert
    random graph model} $G(n,p)$ is a discrete probability space over
  $G^n$ with event space $\mathcal{P}(G^n)$ and probability measure
  $P$, such that every graph $G$ with $\abs{E(G)}=k$ edges appears
  with equal probability%
  \[%
    P(G) = {p^k(1-p)^{n(n-1)-k}},%
  \]%
  for $0 \leq k \leq n(n-1)$. 
\end{definition}

\begin{remark}Clearly $G(n,p)$ is well-defined, as there exist $\binom{n(n-1)}{k}$
distinct labeled graphs with $n$ vertices and $k$ edges and thus 
\[
  \sum_{G \in G^n} P(G) =  \sum_{k=0}^{n(n-1)}  \binom{n(n-1)}{k}
  p^k(1-p)^{n(n-1)-k} = 1, % = (p+(1-p))^{n(n-1)}
\]
after the binomial theorem.
% Which sigma algebra -> power set (discrete probability space)
\end{remark}


Equivalently, the Gilbert random graph model can be defined as a
stochastic process;
%------------------------------------------------
\marginpar{equivalent definition as random process} 
%------------------------------------------------ 
to an empty graph with $n$ vertices, for each of the $n(n-1)$ vertex
pairs an edge is added at random and independently with probability
$p$. The probability to obtain a specific graph $G$ with $k$ edges is
then obviously $p^k(1-p)^{n(n-1)-k}$, already proofing the
equivalence, since assuming a process as above with edge probability
$p'$ such that the induced probability measure on $G^n$ equals
$P$ from \ref{def:gilbert_random_graph}, already yields $p = p'$ in
the choice of $n=2$ and $k=1$. 

\begin{proposition}
  In- and out-degree distribution of vertices in the Gilbert random
  graph model are binomial.
\end{proposition}
%
\begin{proof}
  Let $X$ be a random variable on the random graph model, mapping to
  the in-degree (out-degree) $d_G(v)$ of a vertex $v$ of a graph $G
  \in G^n$. There are $n-1$ other vertices that, with probability $p$,
  project to $v$ (receive input from $v$), thus
  \[
    P(X=k) = \binom{n-1}{k} p^k (1-p)^{n-1-k},%
  \]%
  % $X: G^n \to \mathbb{R}, G \to |E(G)|$, discrete random variable,
  % with probability distribution $\operatorname{B}_{n(n-1),p}$.
  % \[
  % \operatorname{B}_{n(n-1),p}(k) = \binom{n(n-1)}{k} p^k(1-p)^{n(n-1)-k}
  % \]
  showing that $P^X = \mathcal{B}_{n-1,p}$.
\end{proof}

The Gilbert random graph model is therefore also often referred to as
\textit{binomial random graph}\index{binomial random
  graph}. %\parencite{Janson_Random-graphs}
As typical neuronal networks are large ($n \geq 10^3$) with sparse
connectivity ($p \approx 0.1$), in- and out-degree distribution can be
approximated by a Poisson distribution, $P^X(k) \approx
\operatorname{Pois}_{\lambda}(k)$, with $\lambda = (n-1)p$, after the
Poisson limit theorem.

Most results in the study of random graph models consider $n\to
\infty$. In this study we are mostly interested in patterns of
connectivity that arise in local circuits, leaving behind limit
considerations and employ the Gilbert random graph model as a
reference for the development of more detailed and specific random
graph models.


%%% Local Variables: 
%%% mode: latex
%%% TeX-master: "../dplths_document"
%%% End: 
