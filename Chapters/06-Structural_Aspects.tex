%************************************************
\chapter{Structural Aspects}\label{ch:structural_aspects} % $\mathbb{ZNR}$
%************************************************
Subjecting the anisotropic network model to a critical
examination of its structural features, we identify prevalent
patterns of connectivity and relate theoretical and computational
results to findings from experiments in the rat's visual cortex.

\section{Introduction}


% \setlength{\textfloatsep}{40.0pt plus 2.0pt minus 4.0pt}

% \addtolength{\parskip}{\baselineskip} %Absätze im Text werden auch tatsächlich zu Absätzen%
% \parindent 0pt

             

Investigation \marginpar{
    \begin{center}
      \includegraphics[width=0.9\linewidth]{img/HCP_text_logo.png}
    \end{center} \vspace{-0.3cm}
    \mbox{\textrm{\href{http://www.humanconnectome.org/}{humanconnectome.org}}}}
  of the brain's connectivity is an ongoing endeavour.  Concurrent
  collaborative efforts like the Human Connectome Project
  [\textcolor{linkgrey}{HCP}]%??
  , the Open Connectome Project [\textcolor{linkgrey}{OCP}]%??
  and the Allen Brain Atlas [\textcolor{linkgrey}{ABA}]%??
  , intent on mapping the 'wiring' of the brain, as well as the
  continued development of experimental techniques and computational
  resources, demonstrate \marginpar{Open Connectome Project
    \href{http://www.openconnectomeproject.org/}{openconnectomeproject.org}}the
  great interest in advancing this field.

Research in brain connectivity spreads over the whole scale \marginpar{\begin{center}
    \includegraphics[width=1.\linewidth]{img/AllenBrainLogo.png}\end{center}\vspace{-0.3cm}\mbox{\textrm{\href{http://www.brain-map.org/}{brain-map.org}}}} of the
brain; from the mapping of fiber pathways between brain regions at the
macroscopic level, to the synaptic connections of individual neurons
on the microscale, researchers are trying to identify the links that
enable the brain its characteristic cognitive abilities.
%Macroscale: Can cite Sporns2004
In the search for structural connections, these links are of
anatomical nature. However, statistical dependencies and causal
relationships between the distinct computational units in the brain
are being researched with equal emphasis \parencite{Scholarpedia-BrainConnectivity}.

Connectivity in the context of the directionally heterogenous
geometric networks introduced %?? wording
 in Section~\ref{sec:anisotropic_network_model}, refers in this chapter to
structural links. So far, we have only briefly mentioned that the
network's nodes should be interpreted as individual neurons; to allow
for a discussion of functional relationships between nodes, we have yet
to provided a physical description of a neuron's function. As such, we
will here explore the network's structural connectivity, modeling
synaptic contacts between axon and dendrites of individual neurons.


\subsection*{Synaptic Connectivity}

In the local cortical circuits the anisotropic geometric model was
derived from, synaptic connectivity is a major mode of configuration.
In those networks, connectivity has been determined to be neither
completely random nor exclusively specific
[\textcolor{linkgrey}{Source}].%??
Recurring patterns of connectivity have been identified by several
reports \parencite{Sporns2004,Song2005,Perin2011}.

The impact of this structural specifity discovered in local networks
is shown to be significant; while the linking of network structure and
network dynamics remains an active field of research, several studies
were able to employ computational and theoretical models to establish
such a connection. A study by Zhao et al. from 2011, for example,
demonstrates how second order connectivity statistics affect a
network's propensity to synchronize
\parencite{Zhao2011}. In the same year, Alex Roxin
reported on the influence of in- and out-degree distributions on
dynamics of neural network \parencite{Roxin2011}. Later,
Pernice et al. were able to link structural connectivity to spike
train correlations in neural networks
\parencite{Pernice2011}.


\subsection*{Mapping synaptic connectivity in experiments}

Experimentally, paired intracellular recordings are used to determine
synaptic connectivity in cortical slices. Using two electrodes, one
inserted in the cell and one outside the cell, a single intracellular
recording allows for measurement of a cell's membrane potential
\parencites[Chapter 3]{Brette_Neural-activity}[]{Scholarpedia-IntracellularRecording}. Simultaneous
recordings from multiple neurons are then able to infer synaptic
connectivity by evoking an action potential through current injection
in one neuron and observing the change of membrane potential in the
other cells \parencite{Song2005}.

\vspace{0.35cm}
\begin{figure}[H]
  \centering
  \includegraphics[width=.75\linewidth]{img/song2005-quadruplet_recordings.png}
  \caption{Song et al. use quadruple whole-cell recordings, observing
    simultaneously the membrane potential of four neurons.
    \textbf{A)} Contrast image showing four thick-tufted L5 neurons
    \textbf{B)} Fluorescent image of the same cells after patching on
    \textbf{C)} Evoking an action potential in the presynaptic neuron
    causes characteristic membrane potential change in the
    postsynaptic neuron \textbf{D)} Infering synaptic connectivity
    from the EPSP waveform observed in C). Image from \parencite{Song2005}.
    % \texttt{\textcolor{linkgrey}{3b056efe-3ebc}}
  }
\end{figure}
\vspace{0.45cm}

While techniques for paired intracellular recordings are rapidly
developing, their ability to capture connectivity patterns of large
networks is yet very limited. To this date, the connectome of
\textit{C. Elegans} remains the outstanding exception of a
connectivity configuration that has been fully mapped
[\textcolor{linkgrey}{Source}]%??
. Even in the state-of-the-art experiment conducted by Perin et al.,
using a setup capable of recording up to twelve neurons
simultaneously, the authors note that an investigation of degree
distribution was not carried out, due to lack of sufficient data
\parencite{Perin2011}.

\subsection*{Exploiting the benefits of a geometrical model}

Working with a geometrical network model and its computational
implementation, such restrictions disappear; the full information
about the network, in form of its connectivity matrix, is given at
point in time and can be easily queried for. Experiments that may take
days to perform \textit{in vivo}, can be completed in a matter of seconds \textit{in
silico}. As such, geometrical models lend themselves to extensive
examination of their structural aspects.

In trying to exploit these advantages, two approaches present
themselves. One may construct a network model that
extrapolates\graffito{Extrapolation vs. reduction} the known
biological configuration; a full structural examination of these
networks could possibly expose relevant patterns not yet observed. For
this approach a sophisticated understanding of the biological
configuration is critical. Neuron morphology, however, is difficult to
describe and extract.

For this analysis we suggest a reductionist approach. Having motivated
an abstract model reflecting a cortical network's directional
heterogeneity, we distinguish emerging patterns of connectivity,
specific to directionally heterogeneous networks, from results, that
only indirectly stem from the network's anisotropy, in the hopes to be
able to characterize the significance of directional heterogeneity in
structural connectivity of cortical circuits.


\subsection*{Structural aspects of the heterogeneous model}

In this chapter we subject the anisotropic network
model introduced in Section~\ref{sec:anisotropic_network_model} to a
critical analysis of its structural aspects. General network topology,
as well as specific modes and patterns of connectivity, are to be
identified and laid out for comparison with findings in biological
neural networks.

In an effort to map out structural features that can be directly
associated with the network's directional \marginpar{employing
  anisotropy measure} heterogeneity, it is crucial to differentiate
such findings from results that are only indirectly caused by the
network's anisotropy. To this end, already in
Section~\ref{sec:anisotropy_measure} we developed a measure to
quantify the degree of anisotropy prevalent in a given network;
throughout this chapter we will now frequently employ this measure to
determine which structural aspects are originating from the network's
heterogeneity, and which aspects are to be attributed solely to the
network's distance dependency.

Accordingly, results from this investigation are categorized in two
sections: The first section, 'Section 2' %??
, describes structural aspects that can not be directly attributed to
the model's anisotropy. The second section, 'Section 3' %??
, then presents results that are truly features of network's
directional heterogeneity.


% ######################################################################### %
% ------------------------------------------------------------------------- %
%                             Comment Stuff
% ------------------------------------------------------------------------- %
% ######################################################################### %


% Human Connectome Project and the Open Connectome Project, the
% continued development of experimental techniques and of resources
% like the Brain Connectivity Toolbox software, as well as the
% research of theoreticians and experimentalists, are all dedicated
% towards the common goal of charting brain connectivity.

% In this general sense, brain connectivity can refer to linking
% between distinct units at various scales. From the mapping of fiber
% pathways between brain regions on the macroscale, to the synaptic
% connections of individual neurons on the microscale,
% [\textcolor{linkgrey}{Scholarpedia}].

% It is interesting not only to investigate for anatomical
% connections, but functional and causal as well. However, exploring
% the aspects of our specific geometric network model, in this chapter
% the connectivity of interest to us is the structural connectivity at
% the microscale, that is synaptic connections between individual
% neurons.

% \marginpar{Human Connectome Project
% \mbox{\url{humanconnectome.org}}}

% \parbox{1.8cm}{Human Connectome
% Project}\includegraphics[width=0.45\linewidth]{img/HCP_logo.png}
% \mbox{\url{humanconnectome.org}}}


% Ho Ko connectivity -> specific dynamics.
% Pernice -> Spike Train Correlations. Maybe Shepherd 2005?

% ...and have been linked to brain function and dynamics.

% *What cool things synaptic connectivity does.  , stores memory, drives
% dynamics, etc.  While brain is plastic, there is structure that is
% believed to provide a framework, boundary conditions to

% *patterns of synaptic connectivity is neither completely random nor
% exclusively specific, patterns emerge [Sporns, Perin, Song]
 
% This specific, non-random connectivity largely impact dynamics and
% brain function:


% Connectivity in the directionally heterogenous geometric networks
% introduced in ??, models synaptic contacts between axon and
% dendrites of individual neurons. In this chapter

% It is then the task of this theoretical framework to provide results
% interesting to the biological situation. An investigation on how well
% an introduced model can reproduce certain structural aspects of
% networks that have already been fund is integral to the study of a
% computational model. But, furthermore, a model should aspire to
% extrapolate results found in the biological
% situation.





% ######################################################################### %
% ------------------------------------------------------------------------- %
%                     Degree distribution
% ------------------------------------------------------------------------- %
% ######################################################################### %

\section{Degree distribution}\label{sec:degree_distribution}

The in- and out-degree of vertex in a directed graph describes the
number of incoming and outgoing connection from and to other vertices
(cf. Definition~\ref{def:in_out_degree}). As a fundamental concept in
graph and network theory, the degree distribution is integral in the
categorization of networks and allows for the estimation of graph
properties.

Degree distribution was shown to have strong impact on the dynamics of
neuronal networks models commonly used in computational neuroscience
research \parencite{Roxin2011}. Increasing in-degree variance for
example could be connected to the appearance of oscillations in the
network. Extracting degree distributions from biological networks
however, remains a challenge as many neurons need to be tracked
simultaneously to obtain enough data to confidently estimate degree
distributions. 

\begin{figure}[H]
  \centering
  \includegraphics[width=0.7\textwidth]{%
    plots/9326138e.pdf}
  \caption{\textbf{In-degree distribution in anisotropic networks
      shows comparably high variance and is skewed to the left} From 250
    anisotropic networks in-degree distributions were extracted and
    are shown in a normed histogram plot, errorbars SEM. Comparison with the
    binomial degree distribution (red) of a Gilbert random graph model
    with matching parameter set ($N=1000$, $p =0.116$) shows higher
    variance of in-degrees in anisotropic networks (sample variance $=
    344.54$, variance of binomial distribution $Np(1-p) = 102.44$.)
    Skewness to the left of the sample is $-0.1763.$
    (\smtcite{9326138e})}
  \label{fig:in_degree_ER_compare}
\end{figure}

Here we analyze in- and out-degrees in the anisotropic network
model. First we find that compared to the binomial in-degree
distribution of a Gilbert random graph model, in-degrees of vertices
in anisotropic networks display higher variance and their distribution
is skewed to the left (\autoref{fig:in_degree_ER_compare}). However,
this specific in-degree profile is not an intrinsic property of
anisotropy, as the distribution remains stable under manipulation of
the anisotropy degree and closely matches the profile of a purely
distance-dependent network (\autoref{fig:in_degree_rewiring}). This
result agrees with findings of \textcite[Fig. S3]{Perin2011}, who were
able to recreate degree distributions from their experiment with layer
V thick-tufted pyramidal cells in neonatal rats from the extracted
distance-dependent connection profiles alone.



\begin{figure}[H]
  \centering
  \renewcommand{\tabcolsep}{2pt}
  \setlength\extrarowheight{0pt}
  \begin{tabular}{lll}
    \begin{overpic}[width=0.28\textwidth]{%
        plots/77995b6b_in000.pdf}
      \put(12,56){\small $\eta = 0$}
    \end{overpic}
    &
    \begin{overpic}[width=0.28\textwidth]{%
        plots/77995b6b_in025.pdf}
      \put(12,56){\small $\eta = 0.25$}
    \end{overpic}
    &
    \begin{overpic}[width=0.28\textwidth]{%
        plots/77995b6b_in050.pdf}
      \put(12,56){\small $\eta = 0.5$}
    \end{overpic}
    \\
    \begin{overpic}[width=0.28\textwidth]{%
        plots/77995b6b_in075.pdf}
      \put(12,56){\small $\eta = 0.75$}
      \put(4,-4){\small$0$}\put(78,-4){\small$200$}
    \end{overpic}
    &
    \begin{overpic}[width=0.28\textwidth]{%
        plots/77995b6b_in100.pdf}
      \put(12,56){\small $\eta = 1$}
      \put(4,-4){\small$0$}\put(78,-4){\small$200$}
    \end{overpic}
    & 
    \begin{overpic}[width=0.28\textwidth]{%
        plots/77995b6b_indst.pdf}
      \put(12,56){\small distance}
      \put(4,-4){\small$0$}\put(78,-4){\small$200$}
    \end{overpic}
    \\
  \end{tabular}
  \caption{\textbf{In-degree distribution not affected by varying
      degrees of anisotropy} In-degree distributions from the 25
    sample graphs (ref ??) and their rewiring stages are plotted in
    normed histograms and listed from rewiring factor $\eta =0$
    (original anisotropic) to $\eta = 1$ (completely rewired, maximal
    isotropy). Comparison shows that varying degrees of anisotropy do
    not influence the degree distribution, in fact in-degree
    distributions match with the degree distribution of an equivalent
    distance-dependent network shown bottom-right (\smtcite{77995b6b}). }
  \label{fig:in_degree_rewiring}
\end{figure}


While the out-degree distribution of vertices in the anisotropic
network also shows itself stable under rewiring, its distribution is
drastically different from the out-degree distribution in a comparable
distance-dependent network (\autoref{fig:out_degree_rewiring}). The
asymmetric, long-tailed distribution is identified as an artifact of
the anisotropic network's spatial confinement; a neuron, closely
located near a surface edge, might have an axon projection out of the
square causing minimal out-degree or, projecting through the entire
length of the surface, may have maximal out-degree. Approximating the
expected number of outgoing connections for a vertex in an anisotropic
network of size $N$, side-length $s$ and axon width $w$ as
\[
  N \frac{w l}{s^2},
\]
with parameters $N = 1000$ and $\frac{w}{s} = 0.252$, we obtain an
upper bound for the expected out-degree, 
\[
  N \frac{w l}{s^2} \leq N\frac{w}{s} \sqrt{2} \approx 350.
\]
If $f(l)$ is the probability density function to find axon length $l$
for a random node $v$ in the anisotropic network model, the out-degree
distribution is then approximated by
%
\begin{align}\label{eq:axon_length_approx}
  \mathbf{Pr}[d_{\mathrm{out}}(v) = N \frac{w l}{s^2}] = f(l),  
\end{align}
%
see also \autoref{fig:out_degree_ER_compare}.

\begin{figure}[H]
  \centering
  \renewcommand{\tabcolsep}{2pt}
  \setlength\extrarowheight{0pt}
  \begin{tabular}{lll}
    \begin{overpic}[width=0.28\textwidth]{%
        plots/77995b6b_out000.pdf}
      \put(12,56){\small $\eta = 0$}
    \end{overpic}
    &
    \begin{overpic}[width=0.28\textwidth]{%
        plots/77995b6b_out025.pdf}
      \put(12,56){\small $\eta = 0.25$}
    \end{overpic}
    &
    \begin{overpic}[width=0.28\textwidth]{%
        plots/77995b6b_out050.pdf}
      \put(12,56){\small $\eta = 0.5$}
    \end{overpic}
    \\
    \begin{overpic}[width=0.28\textwidth]{%
        plots/77995b6b_out075.pdf}
      \put(12,56){\small $\eta = 0.75$}
      \put(4,-4){\small$0$}\put(78,-4){\small$350$}
    \end{overpic}
    &
    \begin{overpic}[width=0.28\textwidth]{%
        plots/77995b6b_out100.pdf}
      \put(12,56){\small $\eta = 1$}
      \put(4,-4){\small$0$}\put(78,-4){\small$350$}
    \end{overpic}
    & 
    \begin{overpic}[width=0.28\textwidth]{%
        plots/77995b6b_outdst.pdf}
      \put(52,56){\small distance}
      \put(4,-4){\small$0$}\put(78,-4){\small$350$}
    \end{overpic}
    \\
  \end{tabular}
  \caption{\textbf{Out-degree distribution not affected by varying
      anisotropy but highly different from distance-dependent
      networks} Out-degree distributions from the 25 sample graphs
    (ref ??) and their rewiring stages are plotted in normed
    histograms and listed from rewiring factor $\eta =0$ (original
    anisotropic) to $\eta = 1$ (completely rewired, maximal
    isotropy). While varying degrees of anisotropy do not influence
    the degree distribution, the characteristic out-degree profile is
    drastically different from the distribution found in equivalent
    distance-dependent networks (\smtcite{77995b6b}). }
  \label{fig:out_degree_rewiring}
\end{figure}

\vspace{-1cm}
\begin{figure}[H]
  \centering
  \includegraphics[width=0.7\textwidth]{%
    plots/019555b0.pdf}
  \caption{\textbf{Characteristic out-degree distribution as an
      artifact of network's boundaries} From 250
    anisotropic networks out-degree distributions were extracted and
    are shown in a normed histogram plot, errorbars SEM. The
    characteristic distribution is identified as an artifact of the
    network's spatial confinement; using
    equation~\ref{eq:axon_length_approx} the out-degree profile is
    approximated (red) by the distribution of axon lengths in the
    anisotropic network (\smtcite{019555b0}).}
  \label{fig:out_degree_ER_compare}
\end{figure}






% ######################################################################### %
% ------------------------------------------------------------------------- %
%                        Small World Properties
% ------------------------------------------------------------------------- %
% ######################################################################### %


\section{Small World Properties}

Sporns papers



\section{Motifs}




%%% Local Variables: 
%%% mode: latex
%%% TeX-master: "../dplths_document"
%%% End: 
