% ************************************************
\chapter{Biology of Neural Networks}\label{ch:Biology} 
% ************************************************


The fundamental computational units in brain networks are
neurons\index{neuron}, electrically excitable cellular elements that
process and transmit information by a cell type dependent regime of
electrical and chemical signals. Neurons are linked through
synapses\index{synapse}, forming together an expansive, interconnected
network of different neuron types, dividing into functionally and
anatomically distinct areas. The number of neurons in the average
human brain is estimated at about 86 billion, connected by
$10^{14}$ - $5\times10^{15}$ synapses \parencite{Herculano2009,
  Drachman2005}. Among the different brain areas studied, the
multilayered cerebral cortex\index{cortex} stands out as a region of
particular interest with many studies analyzing its structural and
dynamical features.

The principal excitatory neuron type in cortical
networks\index{cortical network} are pyramidal cells\index{pyramidal
  cell}. Connection between those neurons are mainly of chemical
nature, in the synaptic contacts between cells the release and
consequent reception of neurotransmitters transmits electrical
signals. While cortical networks are considered sparse, pyramidal
cells typically receive tens of thousands excitatory and several
thousand inhibitory inputs, making up for an overall connectivity of
about $10\%$ in local networks \parencite{Spruston2009}. Such synaptic
contacts are inherently asymmetric; signals travel from the cell body
of a neuron along the axon to be transmitted at a synapse contacting
the dendritic tree of the post-synaptic neuron. Morphology of axon and
dendrite are characteristically different; it is this difference that
is taken up in this study and serves as a basis for the network model introduced in Chapter~\ref{ch:network_model}.

To enable we introduce...
For Brain networks... They are well presented by the mathematical
object of a directed graph, which will be discussed in detail in the
following chapter.

%%% Local Variables: 
%%% mode: latex
%%% TeX-master: "../dplths_document"
%%% End :
