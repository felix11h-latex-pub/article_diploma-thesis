


% ######################################################################### %
% ------------------------------------------------------------------------- %
%                             Directed Graphs 
% ------------------------------------------------------------------------- %
% ######################################################################### %


\section{Directed Graphs}\label{sec:directed_graphs}


Here we introduce the various categories of directed graphs. The main
reference for this section is \textcite{Bang-Jensen_Digraphs}, for the
formal definition below however, we
follow \textcite{nLab}.

\begin{definition}[Directed graphs]
  \index{directed graph} \index{simple graph} \index{pseudograph}
  \index{multigraph} \index{loop graph}
  \label{def:directed_graphs} \index{vertex set} \index{edge set}
  \marginpar{\vspace{0.550cm}\\short:\\$V(G)$ vertex set $E(G)$ edge set}
  A \textit{directed pseudograph} $G$ consists  of two finite
 sets
  %
  % Non-empty? -> No. Empty graph allowed for random 
  % graph process that adds edges and or vertices
  % 
  $V(G)$, the \textit{set of vertices}  of $G$, and $E(G)$, the \textit{set of
    edges} of $G$, and two maps $ s,t: E \to V, $ the \textit{source}
  and \textit{target functions} of $G$. A \textit{directed multigraph}
  is a directed pseudograph without loops, that is the map $d =
  (s,t):E \to V^2$ already maps maps to $V^2\setminus\Delta_V$, where
  $V^2 = V \times V$ denotes the cartesian product and $\Delta_V =
  \{(x,x) \mid x \in V\} \subseteq V^2$ the diagonal. Similarily, a
  \textit{directed loop graph} is a directed pseudograph where $d$ is
  injective. Finally, a \textit{simple directed graph} can be defined
  as a directed pseudograph where $d$ is both injective and already
  maps to $V^2\setminus\Delta_V$.
\end{definition}

Thus, in simple directed graphs, neither parallel edges nor loops
(edges between the same vertex) are allowed, whereas directed
multigraphs and directed loop graphs admit one of them
respectively. We refer to any of the four graph types simply as a
directed graph and only specify the type when needed.



% Given a directed graph $G$, we denote with $V(G)$ the set of vertices
% of $G$ and call it the \textbf{vertex set} of $G$. Analogously, the
% \textbf{edge set} $E(G)$ of $G$ denotes the set of edges of $G$. This
% means, for a directed graph specified as $G = (V,E,s,t)$, we
% have $V(G) = V$ and $E(G) = E$.

A \textbf{morphism} $\phi: G \to H$, between two directed graphs $G=(
V_G, E_G, \allowbreak s_G, \allowbreak t_G)$ and
$H=(V_H,E_H,s_H,t_H)$, consists of a pair of maps $\phi_V: V_G \to
V_H$ and $\phi_E: E_G \to E_H$, such that
\[
s_H \circ \phi_E = \phi_V \circ s_G \mathrm{\quad and \quad} t_H \circ
\phi_E = \phi_V \circ t_G,
\]
that is such that the following diagram commutes:
%
\begin{align*} 
  \xymatrix@+=1.3cm{E_G \ar^{t_G}@<0.66ex>[d] \ar_{s_G}@<-0.66ex>[d]
    \ar^{\phi_E}[r] & E_H \ar^{t_H}@<0.66ex>[d]
    \ar_{s_H}@<-0.66ex>[d]\\ V_G \ar^{\phi_V}[r] & V_H}
\end{align*}
%
A morphism $\varphi: G \to H$, between two directed pseudographs $G$
and $H$ is an \textbf{isomorphism}, if the maps $\varphi_V: V_G \to
V_H$ and $\varphi_E: E_G \to E_H$ are bijections. Two directed
pseudographs are called \textit{isomorphic} if there exists an
isomorphism inbetween them.


\begin{remark}
  The last definition implies that, if there exists an isomorphism
  $\varphi: G \to H$, an isomorphism $\psi: H \to G$ can be
  found. This isomorphism is, of course, easily constructed via
  $\psi_V: V_H \to V_G, v \mapsto \varphi_V^{-1}(v)$, $\psi_E: E_H \to
  E_G, e \mapsto \varphi_E^{-1}(e)$.
\end{remark}

\begin{figure}[H]
  \centering 
  \makebox[0.6\textwidth]{%
    \begin{overpic}[width=0.25\textwidth]{%
        tikz/directed_pseudograph.pdf}
      \put(-15,35){\small\textbf{A}}
    \end{overpic}
    \hfill
    \begin{overpic}[width=0.25\textwidth]{%
        tikz/directed_multigraph.pdf}
      \put(3,35){\small\textbf{B}}
    \end{overpic}
    }%
  \vfill
  \vspace{0.25cm}
  \makebox[0.6\textwidth]{%
    \begin{overpic}[width=0.25\textwidth]{%
        tikz/directed_loopgraph.pdf}
      \put(-15,35){\small\textbf{C}}
    \end{overpic}
    \hfill
    \begin{overpic}[width=0.25\textwidth]{%
        tikz/directed_simple_graph.pdf}
      \put(3,35){\small\textbf{D}}
    \end{overpic}
    }%
    \caption{%
      \textbf{Examples of the directed graph types}
      \textbf{A)} directed pseudograph \textbf{B)} directed
      multigraph \textbf{C)} directed loop graph \textbf{D)} simple
      directed graph.} %??
  \label{fig:directed_graph_types}
\end{figure}

The subobjects in the directed graph category are
subgraphs. Containing only a subset of the vertices of the original
graph, it may contain any number of eligible edges from $G$.

\begin{definition}[Subgraph]\index{subgraph}
  Let $G$ be a directed graph. A \textit{subgraph} $H$ of $G$ is a
  directed graph with vertex set $V(H) \subseteq V(G)$ and edge set
  $E(H) \subseteq E(G)$ with source and target functions from $G$
  restricted to $E(H)$, such that for every $e \in E(H)$ the source
  and target $s(e),t(e)$ are in $V(H)$. We call the subgraph
  \textit{full} if \index{subgraph!full}
  \[
    E(H) = \{e \in E(G) \mid s(e), t(e) \in V(H)\}.
  \]
  that is if $H$ contains all edges of $G$ with source and target in
  $V(H)$.
\end{definition}

A \textbf{motif}\index{motif} $H$ in $G$ is then a full, connected
subgraph of $G$, that is a subgraph such that every vertex in $H$ is
source or target of at least one edge in $E(H)$. An $n$-motif then
refers to a motif $H$ with $\abs{V(H)} = n$. Usually motifs with only
few vertices are considered and are often understood as the
\enquote{building blocks} of graphs. In neuronal networks, specific
connection patterns are associated with specific dynamical
functionality, making motifs an import aspect in the theory of neural
network dynamics and a central object of interest in this thesis.

\begin{definition}[Weighted directed graphs]
  \index{weighted graph}
  An \textit{edge-weighted directed graph} is a directed graph $G$ along
  with a mapping $\omega: E(G) \to \mathbb{R}$, called the
  \textit{weight function}. Similarly, a \textit{vertex-weighted
    directed graph} is a directed graph with a mapping $\nu: V(G) \to
  \mathbb{R}$.
\end{definition}


% \vspace{-0.21cm}
% \definecolor{lightgray}{rgb}{0.88, 0.88, 0.88}
% \begin{center}
%  \setlength{\fboxrule}{0pt}
%  \fcolorbox{black}{lightgray}{
%    \begin{minipage}[c]{0.95\textwidth}

%      \vspace{0.2cm}

%      \setlength{\intextsep}{0pt}%
%      \setlength{\columnsep}{8pt}%

%      \textbf{A directed graph category for neural networks}\\

%      In brain networks, synaptic connections between neuron are
%      inherently directed. While there is some feedback, pre- and
%      postsynaptic neuron are clearly differentiate. Existing
%      connections between neurons are often mediated over several
%      synapses with different synaptic weights which .

%      Therefore . In dynamic two neuron types . 


%      A vertex

%      $\omega: E(G) \to \mathbb{R}^{+}$ \textbf{and}
%      vertex weights $\nu: V(G) \to \{-1, 1\}$. Synaptic weight
%      $\mathrm{syn}(e)$ for edge $e$ is then \[\mathrm{syn}(e) =
%      \nu(s(e))\,\omega(e).\]


%      \vspace{0.2cm}
%    \end{minipage}}
% \end{center}



\begin{remark}  
  A directed graph $G$ can be equivalently defined
  \marginpar{Equivalent definiton for directed loop graphs} as a pair
  of finite sets $V$, the \textit{set of vertices} of $G$, and $E
  \subseteq V^2$ the \textit{set of edges} of $G$.  For an edge $(x,y)
  \in E$, we call $x$ the \textit{source} and $y$ the target of the
  edge $(x,y)$. Source and target functions are then uniquely
  determined as the projections on the first and second component, 
  \[s = \mathrm{pr}_1, t = \mathrm{pr}_2: E(G) \to V.\] Conversely,
  the edge set $E(G) \subseteq V^2$ can be determined from the source
  and target functions as $E:=\{(s(e),t(e)) \mid e \in E\}$. The
  trivial identities $(x,y) = (\mathrm{pr}_1(x,y),\mathrm{pr}_2(x,y))$
  and $\mathrm{pr_1}(s(e), t(e)) = s(e)$ with $\mathrm{pr_2}(s(e),
  t(e)) = t(e)$ quickly verify the equivalence of the definitions.
  Given a directed loop graph $G$, we often assume the graph to be
  given in this form and write edges as $e=(x,y)$. Note that this
  concept is more complicated to introduce for directed pseudographs,
  since parallel edges $e$ and $e'$ should be differentiated in the
  egde set of $G$, establishing the need for $E(G)$ to be a multi- or
  indexed set, notions we are trying to avoid in this document.
\end{remark}



% A given directed loop graph $G = (V,E,s,t)$, can always be
% represented by a canonical isomorphic directed loop graph $G' =
% (V,E')$, where $E':=\{(s(e),t(e)) | e \in E\} \subseteq V^2$. For a
% directed loop graph $G'$ in canonical form, the source and target
% functions $s',t'$ do not need to be specified, since they are uniquely
% determined as the projections on the first and second component, $s' =
% \mathrm{pr}_1, t' = \mathrm{pr}_2$. Wait, not so easy because
% multi-sets, however add synaptic strength of parallel edges to make
% directed loop-graphs! (Bang-Jensen)



From now on any directed graph is assumed to be a directed
loop graph. Although most, if not all, concepts work for directed
pseudographs just as well, we want to start to heavily use the
canonical edge representation, which when talking about pseudograps
makes problems as mentioned before.

For a pair of vertex sets $X,Y \subseteq V(G)$ of a directed graph $G$
we write 
\[ (X,Y)_G = \{(x,y) \in E(G) | x \in X, y \in Y \}\] for the set of
edges with source in $X$ and target in $Y$. \marginpar{Notation for
  target and source sets} \index{target set}\index{source
  set}Specifically we write $T(x) = (x,V(G))_G$ for the set of
\textit{targets} for edges originating from the vertex $x$ and $S(x)
:= (V(G),x)_G$ for the set of \textit{sources} for edges projecting to
$x$.





\begin{definition}[In- and out-degree]  
  \index{in-degree}\index{out-degree}\label{def:in_out_degree}
  For a directed graph $G$ the \textbf{in-degree} $d^-_G(x)$ of a
  vertex $x$ is defined as the number of edges of $G$ with target $x$,
  that is
  \[
  d^-_G(x) = \left|S(x)\right|.
  \]
  Similarily, the \textbf{out-degree} $d^+_G(x)$ of $x$ is defined as
  \[
  d^+_G(x) = \left|T(x)\right|,
  \]
  the number of edges in $G$ with source $x$.
\end{definition}

A basic property of the in- and out-degree in directed graphs is that
number of in-degrees of every vertex, as well the sum of every
out-degree, equal the total number of edges:

\begin{proposition}
  In every directed graph $G$, we have
  \[
  \sum_{x \in V(G)} d^-(x) = \sum_{x \in V(G)} d^+(x) = | E(G) |.
  \]
\end{proposition}

\begin{proof}
  Since $(V(G),x)_G \cap (V(G),y)_G = \emptyset$ for $x \ne y$, we can
  write
  \[
  \sum_{x \in V(G)} d^-(x) = \left| \bigcup_{x \in V(G)} (V(G),x)_G
  \right| = \left| (V(G),V(G))_G \right| = | E(G) |.
  \]
  Analogously for the out-degree.
\end{proof}


%%% Local Variables: 
%%% mode: latex
%%% TeX-master: "../dplths_document"
%%% End: 
