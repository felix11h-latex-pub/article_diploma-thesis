


% ######################################################################### %
% ------------------------------------------------------------------------- %
%                                Rewiring
% ------------------------------------------------------------------------- %
% ######################################################################### %


\section{Rewiring}\label{sec:rewiring}

% In the network configuration introduced in
% section~\ref{sec:network_model} strong directional anisotropy is
% present: Edges originating from one node \enquote{point in the same
%   direction}, that is they connect to other nodes which cluster around
% a. In this section we introduce an algorithm


It is in our highest interest to compare results to. 
%------------------------------------------------
\marginpar{eliminate anisotropy through rewiring}
%------------------------------------------------ 
To this end we introduce an algorithm that preserves
distance-dependent connectivity as found in
Proposition~\ref{distance_prof}, but eliminates anisotropy in network
connectivity by consecutively rewiring existing connections to new
suitable targets.


Rewiring as presented here, will provide the transition from
anistropic connectivity to networks isotropic in connectivity, closely
resembling purely distance-dependent networks. Applying this process
only partially will then allow us to analyse structural features as
they change with a varying degree of isotropy, asserting the
importance of this process to our study.

In designing the specific rewiring algorithm we identify two
requirements that our implementation should satisfy:
\vspace{-6pt}
\begin{enumerate}
  \itemsep-11pt
  \item elimination of anisotropy in connectivity 
  \item preservation of distance-dependent connectivity
\end{enumerate}
\vspace{-6pt}%
The second point is especially important to us, as we want to impose
isotropy on the network at \enquote{minimal cost}, that is by changing
as little as possible about the other characteristics of the network's
connectivity. The following process respects both of the points above:
For every edge between vertices $v$ and $v'$ with inter-vertex
distance $x$, identify neurons with distance to $v$ in the range of
$(x-\varepsilon, x+\varepsilon)$ as potential new targets. Then pick
at random one of these vertices, including $v'$, as a new target for
the current edge, if such an edge doesn't already exist. %??
In the graph theoretic context we formally define rewiring as follows:

\begin{definition}
  Let $G$ be an anisotropic geometric graph with $\abs{V(G)} =
  n$. Then we define a rewiring of $G$ to be probability space over
  $G^n_{\Phi}$, induced by the following process: For every edge $e
  \in E(G)$ uniformly at random pick a potential new target $t'(e)$
  from the set $M_e = T_e \setminus K_e$, where $T_e$ is the set of all
  vertices that differ in their distance to $s(e)$ less than
  $\varepsilon$ from the distance of $s(e)$ to $t(e)$,
  \[ 
  T_e = \left\{v \in V(G) \setminus s(e) \mid \abs{\mathrm{d}(s(e),v)
      - \mathrm{d}(s(e),t(e))} < \varepsilon \right\} %?? definition
                                %of distance d?
  \]
  and $K_e$ the set vertices that already are connected to $s(e)$ by
  another edge, 
  \[
  K_e = \left\{v \in V(G) \mid \exists\, e' \in E(G): s(e') = s(e),
      t(e') =v \right\}.
  \]
\end{definition}

Let $\tilde{C}(x)$ be the distance-dependent connectivity profile of a
rewiring $R_{\varepsilon}$ of an anisotropic graph $G_{n,w}$. Denote
with $C(x)$ the distance-dependent connection probability of the
$G_{n,w}$. We can estimate the
\begin{align*}
  \mathbf{E}[\tilde{C}(x)] - C(X) 
    & = \int_{x-\varepsilon}^{x+\varepsilon} f(x') C(x') \, dx -
        C(x)\\
    & = \frac{1}{2\varepsilon}\int_{x-\varepsilon}^{x+\varepsilon}
        C(x') - C(x) \, dx \\
    & = \frac{1}{2\varepsilon} \left\{ \int_{x-\varepsilon}^{x} C(x') -
        C(x) \, dx - \int_x^{x+\varepsilon} C(x') - C(x) \, dx
        \right\} \\
    & = \frac{1}{2\varepsilon} 
\end{align*}


% for every neuron: 
%    for every outgoing connection:
%        x = distance to target
%        new_targets = all nodes in distance $(x-\epsilon,x+\epsilon)$



\begin{remark}
Partial rewiring. $R_{\varepsilon, \eta}$
\end{remark} 

Here we choose $\varepsilon = ??$.

\vspace{0.2cm}
\begin{figure}[H]
  \centering 
  \makebox[0.875\textwidth]{%
    %\includegraphics[width=0.4\textwidth]{dist_rew_org.pdf}%
    \begin{overpic}[width=0.4\textwidth, frame]{%
        tikz/distance_rewire_L3.pdf}
      \put(2,102){\small{Before}}
    \end{overpic}
    \hfill
    \begin{overpic}[width=0.4\textwidth, frame]{%
        tikz/distance_rewire_L4.pdf}
      \put(2,102){\small{After}}
    \end{overpic} 
  }%
  \caption{\textbf{Rewiring transforms anisotropic geometric graphs to
      networks with isotropic connectivity} For a given edge $e$ with
    a distance $x$ from its source vertex $v$ to its target vertex
    $t(e)$, potential new targets (striped) are found in within a
    distance $(x-\varepsilon, x+\varepsilon)$ of $v$. The rewired edge
    then projects from $v$ to a new target $t'(e)$, randomly chosen
    from the set of vertices within in this range. Inter-vertex
    distance between $v$ and $t'(e)$ differs by less than
    $\varepsilon$ from $x$, ensuring that for small $\varepsilon$ the
    original distance-dependent connectivity is preserved.}
  \label{fig:distance_rewiring}
\end{figure}

 

% \begin{algorithm}
% Let $N(n,e,w) = (G,P,a)$ be  Then 
% \normalfont
% \begin{algorithmic}%[1] <-- gives line numbers
% \For {$v \in V(N_G)$}
%   \For {$e \in E_{\textrm{out}}(v)$}
%      \State $x \gets \norm{N_P(v)-t(e)}$
%      \State $T \gets \{w \in V(N_G) \mid  x-\varepsilon \leq
%      \norm{N_P(v)-N_P(w)} < x+\varepsilon\}$
%      \State $t(e) \gets \textrm{choice} T$
%   \EndFor
% \EndFor

% \end{algorithmic}
% is defined.
% \end{algorithm}


%%% Local Variables: 
%%% mode: latex
%%% TeX-master: "../dplths_document"
%%% End: 
