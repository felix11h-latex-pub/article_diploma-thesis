





% ######################################################################### %
% ------------------------------------------------------------------------- %
%                          Walks and Distances
% ------------------------------------------------------------------------- %
% ######################################################################### %


\section{Walks and distances}\label{sec:walks_and_distances}


Let $G$ be a directed graph \red{(what does it mean here?)}. A
\textbf{walk} $W$ in $G$ is an alternating sequence
$(x_1,e_1,x_2,e_2,x_3,\ldots,x_{n-1},e_{n-1},x_n)$ of of vertices
$x_i$ and edges $e_i$ from $G$, such that
\[
s(e_i) = x_i \quad \mathrm{and} \quad t(e_i) = x_{i+1}, \:\,
\mathrm{for}\, i=1,..,n-1,
\]
that is, such that the vertices are connected by the edges inbetween
them. We denote the set of vertices $(x_1,\ldots,x_n)$ of $W$ as
$V(W)$ and the sequence of edges $(e_1,\dots,e_{n-1})$ as $E(W)$
\red{(need it?)}.

The vertices $x_1$ and $x_n$ are called the \textit{end vertices} of
$W$ and we also say that $W$ is an $(x,y)$-walk. The \textbf{length}
of $W$ is defined as the length of the sequence of edges; a walk
consisting of only one vertex has length zero. \red{ colon, really?}


\begin{definition}[Distance]
  The \textbf{distance} of two vertices $x$,$y$ in a directed graph
  $G$ \red{(means?)}, is defined as the minimum length of an
  $(x,y)$-walk, if any such walk exists, otherwise
  $\operatorname{dist}(x,y)=\infty$. In short,
  \[
  \operatorname{dist}(x,y) = \inf \{|E(W)| \mid
  W\,\mathrm{is}\,(x,y)\mathrm{-walk}\}.
  \]
  \red{$|E(W)|$ is not explained. Necessary?}
\end{definition}

\begin{proposition}
  The distance function $\operatorname{dist}: V(G) \times V(G) \to
  \mathbb{N}$ of a directed graph $G$ satisfies the triangle equality,
  \[
  \operatorname{dist}(x,z) \le \operatorname{dist}(x,y) +
  \operatorname{dist}(y,z), \:\: \mathrm{for}\:\, x,y,z \in V(G).
  \]
\end{proposition}

\begin{proof}
  Let $x,y,z$ be vertices in $G$. If either no $(x,y)$-walk or
  $(y,z)$-walk exists, the inequality holds by definition. Other wise,
  let $W$ be an $(x,y)$-walk of minimal length and let $U$ be a
  $(y,z)$-walk of minimal length. Certainly, by concatenating $W$ and
  $U$ we obtain an $(x,z)$-walk of length $|E(W)| + |E(U)| =
  \operatorname{dist}(x,y) + \operatorname{dist}(y,z)$, proofing
  that \[ \operatorname{dist}(x,z) \le \operatorname{dist}(x,y) +
  \operatorname{dist}(y,z).
  \]
\end{proof}





%   \begin{definition}[Neighbour, adjacent] Two \textbf{vertices} $x,y \in
% V(G)$ of $G$ are called \textit{adjacent} or \textit{neighbours} if
% there is an edge between $x$ and $y$, $(x,y) \in E(G)$. Two
% \textbf{edges} $e \neq f$ are \textit{adjacent} if they have an end in
% common.
%   \end{definition}


More to do:

\begin{itemize}
\item summarize category of directed (weighted) pseudographs
\item weights!
\item vertices will also be called nodes and neurons, edges will also
  be connections or synapses.
\item subgraphs
\item \sout{vertex set, edge set $E(G), V(G)$.}
\item $\omega(e)$ is weight, connection strength or synaptic weight
  (as a side remark
\item extend to category of weighted directed pseudographs
  (isomorphisms)
\item path
\item adjacency matrix
\item converses of graph related to opposite category?
\item \sout{in- and out-degree}
\item \sout{triangle inequality for distance, $\mathrm{dist}(x,z) \leq
  \mathrm{dist}(x,y) + \mathrm{dist}(y,z)$}
\end{itemize}



References for this chapter:
\url{http://nlab.mathforge.org/nlab/show/graph},
\url{http://nlab.mathforge.org/nlab/show/quiver}, \parencite{Bang-Jensen_Digraphs}




%%% Local Variables: 
%%% mode: latex
%%% TeX-master: "../dplths_document"
%%% End: 
