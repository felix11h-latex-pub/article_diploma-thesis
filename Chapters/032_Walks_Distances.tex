





% ######################################################################### %
% ------------------------------------------------------------------------- %
%                          Walks and Distances
% ------------------------------------------------------------------------- %
% ######################################################################### %


\section{Network Measures}\label{sec:network_measures}


Let $G$ be a (simple) directed graph. A \textbf{walk} $W$ in $G$ is an
alternating sequence
$(x_1,e_1,x_2,e_2,x_3,\ldots,x_{n-1},e_{n-1},x_n)$ of of vertices
$x_i$ and edges $e_i$ from $G$, such that
\[
s(e_i) = x_i \quad \mathrm{and} \quad t(e_i) = x_{i+1}, \:\,
\mathrm{for}\, i=1,..,n-1,
\]
that is, such that the vertices are connected by the edges inbetween
them. We denote the set of vertices $(x_1,\ldots,x_n)$ of $W$ as
$V(W)$ and the sequence of edges $(e_1,\dots,e_{n-1})$ as $E(W)$.

The vertices $x_1$ and $x_n$ are called the \textit{end vertices} of
$W$ and we also say that $W$ is an $(x,y)$-walk. The \textbf{length}
of $W$ is defined as the length of the sequence of edges; a walk
consisting of only one vertex has length zero.

\begin{definition}[Distance]
  The \textbf{distance} of two vertices $x$,$y$ in a directed graph
  $G$, is defined as the minimum length of an
  $(x,y)$-walk, if any such walk exists, otherwise
  $\operatorname{dist}(x,y)=\infty$. In short,
  \[
  \operatorname{dist}(x,y) = \inf \{|E(W)| \mid
  W\,\mathrm{is}\,(x,y)\mathrm{-walk}\}.
  \]
\end{definition}

\begin{proposition}
  The distance function $\operatorname{dist}: V(G) \times V(G) \to
  \mathbb{N}$ of a directed graph $G$ satisfies the triangle equality,
  \[
  \operatorname{dist}(x,z) \le \operatorname{dist}(x,y) +
  \operatorname{dist}(y,z), \:\: \mathrm{for}\:\, x,y,z \in V(G).
  \]
\end{proposition}

\begin{proof}
  Let $x,y,z$ be vertices in $G$. If either no $(x,y)$-walk or
  $(y,z)$-walk exists, the inequality holds by definition. Other wise,
  let $W$ be an $(x,y)$-walk of minimal length and let $U$ be a
  $(y,z)$-walk of minimal length. Certainly, by concatenating $W$ and
  $U$ we obtain an $(x,z)$-walk of length $|E(W)| + |E(U)| =
  \operatorname{dist}(x,y) + \operatorname{dist}(y,z)$, proofing
  that \[ \operatorname{dist}(x,z) \le \operatorname{dist}(x,y) +
  \operatorname{dist}(y,z).\qedhere\]\end{proof}


Note, however, that although the distance of two vertices displays a
triangle inequality, it does not constitute a metric on a directed
graph $G$ as symmetry fails, in general
\[
\operatorname{dist}(x,y) \neq \operatorname{dist}(y,x).
\]
The average distance or average path length in a directed graph $G$
measures how efficiently of information can be transported in the network.

 \begin{definition}[Average path length] The \textit{average path
     length} of a directed graph $G$ with $V(G)=n$ is defined as
     \[
     l = \frac{1}{n(n-1)} \sum_{x\neq y \in V(G)}
     \operatorname{dist}(x,y). \]
\end{definition}

In practice, vertex pairs with $\operatorname{dist}(x,y) = \infty$,
that is pairs that are not connected by a walk, are disregarded in the
computation and the average path length is determined in the connected
components of the graph, ensuring that $l$ is finite.

The concept of a small-world property in graphs was introduced by
\textcite{Watts1998}. Networks associated with the property are
characterized by a small average path length, while however most nodes are
organized in \enquote{cliques}, connecting to nodes that are
themselves neighbors. A measure capturing this property is the (local)
clustering coefficient:

\begin{definition}[Clustering coefficient] The \textit{clustering
    coefficient} of a vertex $x$ in a directed graph $G$ is defined as
  ratio of realized and possible edges between the neighbors of
  $x$. If $N_x$ is the neighborhood of all vertices reciprocally
  connected to $x$,
  \[
    N_x = \{v \mid v \in T(x) \wedge v \in S(x)\},
  \] 
  then the clustering coefficient is given by
  \[
    \operatorname{clust}(x) = \frac{\abs{(N_x,N_x)_G}}{\abs{N_x}
      (\abs{N_x} -1)}.
  \]
\end{definition}

Small-worldness is then described by a low average path length and
high clustering coefficient, usually considered as the mean of all
vertices. 


%   \begin{definition}[Neighbour, adjacent] Two \textbf{vertices} $x,y \in
% V(G)$ of $G$ are called \textit{adjacent} or \textit{neighbours} if
% there is an edge between $x$ and $y$, $(x,y) \in E(G)$. Two
% \textbf{edges} $e \neq f$ are \textit{adjacent} if they have an end in
% common.
%   \end{definition}



% \begin{itemize}
% \item summarize category of directed (weighted) pseudographs
% \item weights!
% \item vertices will also be called nodes and neurons, edges will also
%   be connections or synapses.
% \item subgraphs
% \item \sout{vertex set, edge set $E(G), V(G)$.}
% \item $\omega(e)$ is weight, connection strength or synaptic weight
%   (as a side remark
% \item extend to category of weighted directed pseudographs
%   (isomorphisms)
% \item path
% \item adjacency matrix
% \item converses of graph related to opposite category?
% \item \sout{in- and out-degree}
% \item \sout{triangle inequality for distance, $\mathrm{dist}(x,z) \leq
%   \mathrm{dist}(x,y) + \mathrm{dist}(y,z)$}
% \end{itemize}






%%% Local Variables: 
%%% mode: latex
%%% TeX-master: "../dplths_document"
%%% End: 
