% ************************************************
\chapter{Introduction}\label{ch:Introduction} 
% ************************************************

Brain network connectivity, the description of links between the
brain's computational units, lies at the heart of many theories trying
to explain the exceptionally diverse and robust functionality of the
brain. As an essential component in the investigation of the emergence
of the brain's unique cognitive abilities, connectivity is often
associated with working memory and the remarkable performance in
various perception tasks. Connectivity is, in its essence, intimately
tied to mathematical concepts. Relying heavily on a graph theoretical
framework for the analysis and discussion of network connectivity,
this aspect of brain research constitutes an exciting and highly
relevant example of applied mathematics.

In theoretical and computational neuroscience, neural network models
are studied as a reflection of brain networks. Research in this field
ranges from the modeling of single neurons to the simulation of
large-scale networks. An area of particular interest lies in the
dynamical aspect of neural networks; picking from a wide range of
neuron models, a modeler can then investigate dynamical properties
such as activity synchronization or oscillation. The standard network
model for such considerations is that of a random graph, imposing
completely random connectivity on the
network \parencite{Brunel2000}. However, results over the last years
have repeatedly shown that local cortical circuits display highly
non-random connectivity features not present in random or even
distance-dependent networks \parencite{Song2005, Perin2011}. It is
unclear how to incorporate such features in the common network models,
as the underlying connection principles inducing the non-random
patterns remain yet to be determined.

% such as underlying remain yet to be identified.  unclear how to
% incorporate. Important to identify underlying rules.

The search for such underlying principles is therefore an ongoing
endeavor \parencite[cf.][]{Klinshov2014}. In an attempt to contribute
to this effort, in this thesis anisotropy in connectivity is discussed
as a morpho\-logy-induced connection principle. Condensing
stereotypical anatomy of pyramidal cells in cortical circuits into a
network model, the proposed connectivity rule may not only provide
insight in the emergence of non-random patterns, but can potentially
be a step towards a network archetype improving on the standard random
network model.

% Connectivity is, in its essence, a purely abstract
% concept and therefore intimately tied to mathematical
% theories. Heavily relying on a mathematical framework for the analysis
% and discussion of network connectivity, this aspect of brain research
% is an exciting example and highly relevant example of applied
% mathematics.

\section{Overview}\label{sec:all_overview}

Following the introduction and this outline, a short overview of the
biological terms frequently appearing throughout this text is given as
reference at the end of this chapter. The central mathematical objects
in this study, various directed graph models, are then introduced and
discussed in detail in Chapter~\ref{ch:Graph_theory}. Building on
these concepts, Chapter~\ref{ch:network_model} introduces the
anisotropic network model as the main object of investigation in this
thesis. Next to an in-depth motivation of the anisotropic connectivity
concept, the chapter also introduces the rewiring of networks and a
measure for anisotropy, laying the groundwork for the analysis of
structural features in anisotropic networks in
Chapter~\ref{ch:structural_aspects}. First investigating standard
network attributes like degree distributions and small-world measures,
analysis of higher order connectivity in the latter part of the
chapter reveals the highly interesting emerging patterns in
anisotropic networks. Closing the structural analysis with a critical
discussion of the obtained results, the last chapter of this work
provides an outlook on how anisotropy in connectivity may influence
dynamical aspects.



\section{Biology of Neural Networks}\label{sec:Biology} 



The fundamental computational units in brain networks are
neurons\index{neuron}, electrically excitable cellular elements that
process and transmit information by a cell type dependent regime of
electrical and chemical signals. Neurons are linked through
synapses\index{synapse}, forming together an expansive, interconnected
network of different neuron types, dividing into functionally and
anatomically distinct areas. The number of neurons in the average
human brain is estimated at about 86 billion, connected by
$10^{14}$ - $5\times10^{15}$ synapses \parencite{Herculano2009,
  Drachman2005}. Among the different brain areas studied, the
multilayered cerebral cortex\index{cortex} stands out as a region of
particular interest with many studies analyzing its structural and
dynamical features.

The principal excitatory neuron type in cortical
networks\index{cortical network} are pyramidal cells\index{pyramidal
  cell}. Connection between those neurons are mainly of chemical
nature, in the synaptic contacts between cells the release and
consequent reception of neurotransmitters transmits electrical
signals. While cortical networks are considered sparse, pyramidal
cells typically receive tens of thousands excitatory and several
thousand inhibitory inputs, making up for an overall connectivity of
about $10\%$ in local networks \parencite{Spruston2009}. Such synaptic
contacts are inherently asymmetric; signals travel from the cell body
of a neuron along the axon to be transmitted at a synapse contacting
the dendritic tree of the post-synaptic neuron. Morphology of axon and
dendrite are characteristically different; it is this difference that
is taken up in this study and serves as a basis for the network model
introduced in Chapter~\ref{ch:network_model}. 

To enable the definition and critical discussion of such models, the
next chapter introduces in detail the underlying mathematical concepts
of network connectivity. 









%%% Local Variables: 
%%% mode: latex
%%% TeX-master: "../dplths_document"
%%% End: 
