% ************************************************
\chapter{Introduction}\label{ch:Introduction} 
% ************************************************

Brain network connectivity, the description of links between the
brain's computational units, lies at the heart of many theories trying
to explain the exceptionally varied and robust functionality of the
brain. As an essential component in the investigation of the emergence
of its unique cognitive abilities, brain connectivity is associated
with memory and the performance of many tasks. Connectivity is, in its
essence, a purely abstract concept and is thus intimately tied to
mathematical theories. Providing the concepts mathematical concept
spur reaserach in brain,borrowing many of the existenting to research.

In the field of theoretical and computational neuroscience neural
network models are studied as of brain networks. Studies interested in
the dynamical aspect for example investigate . The standard model for
such simulations is that of random
graph \parencite{Brunel2000}. However, results over the last years
show that local cortical circuits display highly non-random
connectivity features \parencite{Song2005, Perin2011}. It is unclear
how to incorporate such as underlying remain yet to be identified.
unclear how to incorporate. Important to identify underlying rules.

The search underlying principles has
since \parencite{Klinshov2014}. In an effort to contribute to this
discussion, we here investigate anisotropy in connectivity. Motivated
from observations of stereotypical morphology, the may not only
provide but further can be a first step towards network models.

\section{Overview}\label{sec:all_overview}

Following the introduction and this outline, a short overview of the
biological terms frequently appearing throughout this text is given as
reference at the end of this chapter. The central mathematical objects
in this study, various directed graph models, are then introduced and
discussed in detail in Chapter~\ref{ch:graph_theory}. Building on
these concepts, Chapter~\ref{ch:network_model} introduces the
anisotropic network model as the main object of investigation in this
thesis. Next to an in-depth motivation of the anisotropic connectivity
concept, the chapter also introduces the rewiring of networks and a
measure for anisotropy, laying the groundwork for the analysis of
structural features in anisotropic networks in
Chapter~\ref{ch:structural_aspects}. First investigating standard
network attributes like degree distributions and small-world measures,
analysis of higher order connectivity in the latter part of the
chapter reveals highly interesting and spurs a differentiated
discussion on the end of the chapter. Giving an outlook in
Chapter~\ref{ch:dynamical_aspects}, del fin.



\section{Biology of Neural Networks}\label{sec:Biology} 



The fundamental computational units in brain networks are
neurons\index{neuron}, electrically excitable cellular elements that
process and transmit information by a cell type dependent regime of
electrical and chemical signals. Neurons are linked through
synapses\index{synapse}, forming together an expansive, interconnected
network of different neuron types, dividing into functionally and
anatomically distinct areas. The number of neurons in the average
human brain is estimated at about 86 billion, connected by
$10^{14}$ - $5\times10^{15}$ synapses \parencite{Herculano2009,
  Drachman2005}. Among the different brain areas studied, the
multilayered cerebral cortex\index{cortex} stands out as a region of
particular interest with many studies analyzing its structural and
dynamical features.

The principal excitatory neuron type in cortical
networks\index{cortical network} are pyramidal cells\index{pyramidal
  cell}. Connection between those neurons are mainly of chemical
nature, in the synaptic contacts between cells the release and
consequent reception of neurotransmitters transmits electrical
signals. While cortical networks are considered sparse, pyramidal
cells typically receive tens of thousands excitatory and several
thousand inhibitory inputs, making up for an overall connectivity of
about $10\%$ in local networks \parencite{Spruston2009}. Such synaptic
contacts are inherently asymmetric; signals travel from the cell body
of a neuron along the axon to be transmitted at a synapse contacting
the dendritic tree of the post-synaptic neuron. Morphology of axon and
dendrite are characteristically different; it is this difference that
is taken up in this study and serves as a basis for the network model introduced in Chapter~\ref{ch:network_model}.

To enable we introduce 
For Brain networks . They are well presented by the mathematical
object of a directed graph, which will be discussed in detail in the
following chapter.









%%% Local Variables: 
%%% mode: latex
%%% TeX-master: "../dplths_document"
%%% End: 
