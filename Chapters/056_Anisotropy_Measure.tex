


% ######################################################################### %
% ------------------------------------------------------------------------- %
%                           Anisotropy Measure
% ------------------------------------------------------------------------- %
% ######################################################################### %


\section{Anisotropy Measure}\label{sec:anisotropy_measure}

In the last section a method to rewire an anisotropic geometric graph,
such that was introduced. From an . In this chapter we
introduce.. capturing ..

The $G_{n, \Phi}$ be a geometric graph. Then, for every is the
\textit{preferred direction} and its length is \index{preferred direction}

\textcite{Mardia_Directional-statistics}
 
\begin{figure}[H]
\caption{\textbf{illustrate varying levels of anisotropy}}
\end{figure}


\begin{figure}[H]
  \centering
  \makebox[0.75\textwidth]{%
    \renewcommand{\tabcolsep}{2pt}
    \setlength\extrarowheight{0pt}
    \begin{tabular}{ccc}
      \begin{overpic}[width=0.28\textwidth]{%
          plots/038f5b8c.pdf}
         \put(10,57){\small $\eta = 0$}
      \end{overpic}
      &
      \begin{overpic}[width=0.28\textwidth]{%
          plots/9893ab1f.pdf}
        \put(10,57){\small $\eta = 0.25$}
      \end{overpic}
      &
      \begin{overpic}[width=0.28\textwidth]{%
          plots/0c429f3c.pdf}
        \put(10,57){\small $\eta = 0.5$}
      \end{overpic}
      \\
      \begin{overpic}[width=0.28\textwidth]{%
          plots/400a41bc.pdf}
        \put(10,57){\small $\eta = 0.75$}
        \put(4,-4){$0$}\put(90,-4){$1$}
      \end{overpic}
      &
      \begin{overpic}[width=0.28\textwidth]{%
          plots/a5f54cb4.pdf}
        \put(10,57){\small $\eta = 1$}
        \put(4,-4){$0$}\put(90,-4){$1$}
      \end{overpic}
      & 
      \begin{overpic}[width=0.28\textwidth]{%
          plots/a5b674f0.pdf}
        \put(11,57){\small distance-}
        \put(11,45){\small dependent}
        \put(4,-4){$0$}\put(90,-4){$1$}
      \end{overpic} 
    \end{tabular}
  }
  %
  \caption{\textbf{Isotropy distribution changes significantly through
    rewiring} }%?? sumatra label
  \label{fig:anisotropy} %??
\end{figure}




%%% Local Variables: 
%%% mode: latex
%%% TeX-master: "../dplths_document"
%%% End: 
