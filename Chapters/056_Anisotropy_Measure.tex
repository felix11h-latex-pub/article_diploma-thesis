


% ######################################################################### %
% ------------------------------------------------------------------------- %
%                           Anisotropy Measure
% ------------------------------------------------------------------------- %
% ######################################################################### %


\section{Anisotropy Measure}\label{sec:anisotropy_measure}

A measure for anisotropy in connectivity in geometric graphs can be
adapted from concepts in the field of directional statistics. The
\textit{mean direction} of a set of unit vectors in $\mathbb{R}^2$ is
the sum of the vectors, divided by the cardinality of the
set \parencite[cf.][]{Mardia_Directional-statistics}. The resulting
vector gives a preferred direction of the sample, pointing towards a
clustering of the set. The length of the mean direction measures the
vectors dispersion, with a length of close to one implying tight
clustering. Here we adapt the length of the mean direction to provide
a measure of anisotropy in connectivity for each vertex in a geometric
graph:


\begin{definition}[Anisotropy degree]\index{anisotropy
    measure}\index{anisotropy degree}
Let $G_{n, \Phi}$ be a geometric graph. For each vertex $v$ in $G_{n,
  \Phi}$, the anisotropy degree of $v$ is defined as
\[
\lambda(v) =         \frac{1}{\abs{T(v)}} \sum_{w \in T(v)}
\frac{\Phi(w)-\Phi(v)}{\norm{\Phi(w)-\Phi(v)}}
\]
if the target set $T(v)$ of $v$ is non-empty, otherwise we set
$\lambda(v) = 0$.
\end{definition}

By the triangle inequality, the anisotropy degree $\lambda(v)$ takes
values from $0$ to $1$. If targets of $v$ mostly align along a
projection from $v$, the degree $\lambda(v)$ is close to $1$. On
contrast, if targets are widely dispersed, $\lambda(v)$ is almost
$0$. Note however that $\lambda(v) = 0$ does \textit{not} necessarily
imply even distribution of directions, examples for this are easily
constructed \parencite[cf.][]{Mardia_Directional-statistics}.

Considering the distribution of vertex anisotropy degrees in a
geometric graph, the profiles provide an appropriate measure for
anisotropy in connectivity. Recording values for each vertex in
anisotropic networks and their rewired counterparts, we find that
rewiring does indeed eliminate anisotropy
(\autoref{fig:anisotropy_degree}). In fact, distributions in fully
rewired networks resemble anisotropy degrees in distance-dependent
graphs, suggesting that rewired networks can be considered equivalent
to distance-dependent networks in terms of their structural
aspects. This, however, turns out not to be true. In
Section~\ref{sec:degree_distribution} we find strongly varying
out-degree distributions in the two network types. Therefore both
network models remain highly relevant as a reference for the analysis
of structural features in anisotropic networks.

\begin{figure}[H]
  \centering
  \makebox[0.75\textwidth]{%
    \renewcommand{\tabcolsep}{2pt}
    \setlength\extrarowheight{0pt}
    \begin{tabular}{ccc}
      \begin{overpic}[width=0.28\textwidth]{%
          plots/038f5b8c.pdf}
         \put(10,57){\small $\eta = 0$}
      \end{overpic}
      &
      \begin{overpic}[width=0.28\textwidth]{%
          plots/9893ab1f.pdf}
        \put(10,57){\small $\eta = 0.25$}
      \end{overpic}
      &
      \begin{overpic}[width=0.28\textwidth]{%
          plots/0c429f3c.pdf}
        \put(10,57){\small $\eta = 0.5$}
      \end{overpic}
      \\
      \begin{overpic}[width=0.28\textwidth]{%
          plots/400a41bc.pdf}
        \put(10,57){\small $\eta = 0.75$}
        \put(4,-4){$0$}\put(90,-4){$1$}
        \put(45,-5){$\nicefrac{1}{2}$}
      \end{overpic}
      &
      \begin{overpic}[width=0.28\textwidth]{%
          plots/a5f54cb4.pdf}
        \put(10,57){\small $\eta = 1$}
        \put(4,-4){$0$}\put(90,-4){$1$}
        \put(45,-5){$\nicefrac{1}{2}$}
      \end{overpic}
      & 
      \begin{overpic}[width=0.28\textwidth]{%
          plots/a5b674f0.pdf}
        \put(11,57){\small distance-}
        \put(11,45){\small dependent}
        \put(4,-4){$0$}\put(90,-4){$1$}
        \put(45,-5){$\nicefrac{1}{2}$}
      \end{overpic} 
    \end{tabular}
  }
  %
  \captionsetup{skip=15pt}
  \caption{\textbf{Rewiring significantly reduces anisotropy}
    Analyzing vertex anisotropy degrees in the 25 sample graphs
    (Section~\ref{sample_graphs}) and their rewired versions, we find
    that the mean anisotropy degree decrease with increasing rewiring
    factor $\eta$. In completely rewired graphs ($\eta=1$), anisotropy
    degree distribution resembles those of distance-dependent
    networks. (\smtcite{038f5b8c}, \smtcite{9893ab1f},
    \smtcite{0c429f3c}, \smtcite{400a41bc}, \smtcite{a5f54cb4},
    \smtcite{a5b674f0})}
  \label{fig:anisotropy_degree} 
\end{figure}






%%% Local Variables: 
%%% mode: latex
%%% TeX-master: "../dplths_document"
%%% End: 
