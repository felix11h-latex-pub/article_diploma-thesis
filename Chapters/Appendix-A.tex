%*******************************************************
% Appendix
%*******************************************************
% If problems with the headers: get headings in appendix etc. right
%\markboth{\spacedlowsmallcaps{Appendix}}{\spacedlowsmallcaps{Appendix}}




\chapter{Appenidx}





\section{Reproducibility of Computational Results}\label{sec:reproducibility}

Computational implementations of network models and numerical analysis
of their features are an integral part of this study. In order to
ensure the reproducibility of these results, \textit{Sumatra}, a
software for provenance capture of computational projects, was
used \parencite{Sumatra2012}. As a \enquote{lab notebook for
computational researchers}, the software combines version control of
code with the capture of inputs and outputs, as well as the used
parameter sets, for each computational process. Providing a database
and various interfaces, the tool allows not only direct management of
simulations and data but provides a platform to make the computational
reproducible and accessible to others.



In this study, for every computational result referenced \marginpar{on
  the use of Sumatra labels} in the text, a label was cited, referring
to the record of the process in Sumatra's database. Code and produced
data are understood as part of the work of this thesis; giving
references to computational processes in which the results were
obtained, allows the reader at any point to see the exact
implementation of the described process or to look up a parameter not
explicitly mentioned in the text (\autoref{fig:sumatra}). Through this, the work presented
here lives up to of in all areas of science.


Further contributing , with \marginpar{use of free, open source software}single exception of Mathematica, all tools
and software used to create this work 


\newpage

\begin{figure}[H]
  \scriptsize
\hspace{0.4cm}
\begin{minipage}{0.7\linewidth}
\begin{verbatim}
---------------------------------------------------------------------------
Label            : com_inp_N500-trials5-step010
Timestamp        : 2014-06-06 17:29:39.663398
Reason           : common input variance
Outcome          : 
Duration         : 1087.83452988
Repository       : GitRepository at /users/hoffmann/research
Main_File        : comp/common_input_variance.py
Version          : 10690e97199851f08955ed7944f7eadfe71b4bf2
Script_Arguments : com_inp_N500-trials5-step010 <parameters>
Executable       : Python (version: 2.7.3) at /usr/bin/python
Parameters       : rew_frac_min = 0.1
                 : n_trials = 5
                 : rew_margin = 1.25
                 : Torus = False
                 : l_ax = 1000
                 : rew_frac_max = 1.0
                 : N = 500
                 : ed_l = 296
                 : rew_frac_step = 0.1
                 : save_the_graph = True
                 : comp_label = "common_input_variance/"
                 : self_loops_allowed = False
                 : parallel_edges_allowed = False
Input_Data       : []
Launch_Mode      : serial
Output_Data      :[common_input_variance/com_inp_N500-trials5-step010.p
                 : (1aacdd2202ba8ce6a419b6e8cf5d241c2ef37454)]
User             : Felix Hoffmann <Felix11H@github.nomail>
Tags             : clustering
---------------------------------------------------------------------------
\end{verbatim}
\end{minipage}
\normalsize
\captionsetup{skip=18pt}
\caption{\label{fig:sumatra}%
  \textbf{Example Sumatra record entry} Showing the label, parameter
  set and output, Sumatra record entries display the full information
  required to reproduce th. The version number here refers to }
\end{figure}






\section{Mathematica}

% \newgeometry{left=1.0in,right=1.0in}
\vspace{-1cm}
\begin{mathematica}[H]
  \centering
  \captionsetup{format=plain, font=normal, skip=7pt}
  \includegraphics[width=\linewidth, cframe=citegray
  1pt]{mathematica/distance_theorem.pdf}
  \caption{\label{mathematica:distances}%
    Computation of probability density function for distance between
    to random points in square of side length $s$ as supplement to
    proof of Theorem~\ref{theorem:distance_square}. Note that form of
    final result \texttt{Out[7]} differs from solution given in
    \ref{theorem:distance_square} for $1<x<\sqrt{2}$. While proof of
    equivalence could not be achieved analytically, expressions given
    are numerically equivalent.}
  
  \vspace{0.7cm}

  \includegraphics[width=\linewidth, cframe=citegray 1pt]{%
    mathematica/get_motifs_cropped.pdf}
    \captionsetup{format=plain, font=normal, skip=-6pt}
    \caption{\label{mathematica:motif}%
      Computation of three motifs for
      Section~\ref{sec:motifs}. Function \texttt{c[x]} is the
      distance-dependent probability distribution from Theorem and
      \texttt{w[x]} the probability density function for distance
      between two random points in a box
      (cf. Mathematica~\ref{mathematica:distances}, Moltchanov 2012).}

\end{mathematica} 






% ######################################################################### %
% ------------------------------------------------------------------------- %
%                         Supplementary Figures
% ------------------------------------------------------------------------- %
% ######################################################################### %




\section{Supplementary Figures}\label{sec:supp_figures}

%\subsection*{\autoref{ch:network_model}}

\begin{figure}[H]
  \vspace{-1.2cm}
  \centering
  \includegraphics[width=0.8\textwidth]{%
    plots/4afc2727.pdf}
  \captionsetup{skip=-7pt}
  \caption{\label{suppfig:rew_stats}(\smtcite{4afc2727})}

  \vspace{0.5cm}
  \captionsetup{skip=-1pt}
  \includegraphics[width=0.7\textwidth]{%
    plots/c7ee86d7.pdf}
  \caption{\label{suppfig:out_degree}(\smtcite{c7ee86d7})}
  
  \vspace{0.5cm} 
  \includegraphics[width=0.5\textwidth]{plots/064f9b10_apl_1000.pdf}  
  \caption{\label{suppfig:small_world}Average path length for
    anisotropic and distance-dependent networks,
    $N=1000$. (\smtcite{064f9b10})}

  \vspace{0.5cm}
 \captionsetup{skip=2pt}
  \includegraphics[width=0.5\textwidth]{plots/c5f1462b_rew_dist.pdf}
  \caption{\label{suppfig:two_neurons_dist_rew}Probabilities for
    connections in neuron pairs are identical in distance-dependent
    and rewired anisotropic networks. (\smtcite{c5f1462b})}

\end{figure}


% \begin{figure}
%     \renewcommand{\tabcolsep}{1pt}
%     \setlength\extrarowheight{0pt}
%     \begin{tabular}{lll}
%       \begin{overpic}[width=0.33\textwidth]{%
%           plots/5841710e_all_overall}
%         %\put(12,55){\small $\eta = 0$}
%       \end{overpic}
%       &
%       \begin{overpic}[width=0.33\textwidth]{%
%           plots/5841710e_all_single.pdf}
%         %\put(12,55){\small $\eta = 0.25$}
%       \end{overpic}
%       &
%       \begin{overpic}[width=0.33\textwidth]{%
%           plots/5841710e_all_recip.pdf}
%         %\put(12,55){\small $\eta = 0.33$}
%       \end{overpic}
%       \\
%       \begin{overpic}[width=0.33\textwidth]{%
%           plots/5841710e_in_overall.pdf}
%         %\put(12,55){\small $\eta = 0.75$}
%       \end{overpic}
%       &
%       \begin{overpic}[width=0.33\textwidth]{%
%           plots/5841710e_in_single.pdf}
%         %\put(12,55){\small $\eta = 1$}
%         %\put(4,-4){$0$}%\put(78,-4){$200$}
%       \end{overpic}
%       & 
%       \begin{overpic}[width=0.33\textwidth]{%
%           plots/5841710e_in_recip.pdf}
%         %\put(12,55){\small distance}
%         %\put(4,-4){$0$}%\put(78,-4){$200$}
%       \end{overpic}
%       \\
%       \begin{overpic}[width=0.33\textwidth]{%
%           plots/5841710e_out_overall.pdf}
%         %\put(12,55){\small $\eta = 0.75$}
%       \end{overpic}
%       &
%       \begin{overpic}[width=0.33\textwidth]{%
%           plots/5841710e_out_single.pdf}
%         %\put(12,55){\small $\eta = 1$}
%         %\put(4,-4){$0$}%\put(78,-4){$200$}
%       \end{overpic}
%       & 
%       \begin{overpic}[width=0.33\textwidth]{%
%           plots/5841710e_out_recip.pdf}
%         %\put(12,55){\small distance}
%         %\put(4,-4){$0$}%\put(78,-4){$200$}
%       \end{overpic}
%     \end{tabular}
%   \caption{Common neighbor rule for overall connection probability,
%     single connections and reciprocal connections in anisotropic
%     (blue), rewired (red), distance-dependent (green) and tuned
%     anisotropic (orange) networks. (\smtcite{5841710e})} 
%   \label{suppfig:common_neighbor}
%   \end{figure}

% \begin{figure}[H]
%   \centering
%   \includegraphics{

%%% Direct in Chapter.

% \begin{minipage}[t]{0.48\textwidth}
% \begin{align*}
% % \mathbf{P}(X=1) &    =   p_u^3  \\
% % \mathbf{P}(X=2) &    =   6 p_u p_u p_s\\
% % \mathbf{P}(X=3) &    =   3 p_u p_u p_r\\
% \mathbf{P}(X=4) &    =   3 p_s^2 p_u\\
% \mathbf{P}(X=5) &    =   3 p_s^2 p_u\\
% \mathbf{P}(X=6) &    =   6 p_s^2 p_u\\
% \mathbf{P}(X=7) &    =   6 p_s p_u p_r\\
% \mathbf{P}(X=8) &    =   6 p_s p_u p_r\\
% \mathbf{P}(X=9) &    =   3 p_r^2 p_u\\
% \end{align*}
% \end{minipage}%
% \begin{minipage}[t]{0.48\textwidth}
% \begin{equation}
% \label{eq:three_motif_full}
% \begin{aligned}
% \mathbf{P}(X=10) &   =   6 p_s^3   \\
% \mathbf{P}(X=11) &   =   2 p_s^3    \\
% \mathbf{P}(X=12) &   =   3 p_s^2 p_r\\
% \mathbf{P}(X=13) &   =   6 p_s^2 p_r\\
% \mathbf{P}(X=14) &   =   3 p_s^2 p_r\\
% \mathbf{P}(X=15) &   =   6 p_s p_r^2\\
% \mathbf{P}(X=16) &   =   p_r^3 
% \end{aligned}
% \end{equation}
% \end{minipage}


%%% Local Variables: 
%%% mode: latex
%%% TeX-master: "../dplths_document"
%%% End: 