\newpage

\section{Two Neuron Connections}

% Makram 1997, Song 2005. Perin2011

Connectivity in local cortical circuits exhibits a salient feature:
Examining the occurrence of connections in neuron pairs, studies have
repeatedly found that bidirectionally connected neuron pairs appear
much more frequently than expected from the network's overall
connection probability. In layer 5 of the somatsosensory cortex
studies from \textcite{Markram1997_TL5} and \textcite{Perin2011} have
found an overrepresentation of reciprocally connected pairs of thick
tufted pyramidal cells, an observation that has also been reported in
layer 2/3 \parencite{Holmgren2003} and layer 5 \parencite{Song2005} of
the visual cortex. The overrepresentation of bidirectionally connected
pairs is significant, Song et al.~for example found such pairs four times the
expected amount. 

The underlying connection principle imposing this overrepresentation
on the network however remains unclear. Song et al.~discuss the
possibility of known learning rules to explain their findings, leaving
a definitive answer open to further investigation. More recent studies
find overrepresentation of reciprocally connected pairs \textit{in
  vitro} resulting from functional specificity \parencite{Ko2011} and
\textit{in silico} from dense neuron clustering
rules \parencite{Klinshove2014}. 

Is anisotropy in connectivity a possible underlying principle
affecting the occurrence of bidirectional connections?  Here we
investigate whether anisotropy in connectivity can be at cause for the
overrepresentation of reciprocally connected pairs in cortical
circuits.  In random networks, the chance to encounter a specific mode
of connection in a random pair of neurons be can easily be computed
from the overall connection probability $p$. For this let $X$ be the
random variable of the number of edges between two different vertices
in a Gilbert graph $G(n,p)$ with $n \ge 2$. As the edges are
independently realized resulting in a simple directed graph, we have
\begin{equation}
  \label{eq:pairs}
  \begin{aligned}%
    & \mathbf{P}(X=0) = (1-p)^2   &&\text{unconnected pair,}  \\
    & \mathbf{P}(X=1)= 2p(1-p)    &&\text{single connection,}\\
    & \mathbf{P}(X=2) = p^2       &&\text{reciprocal connection};
  \end{aligned}%
\end{equation}
in short $\mathbf{P}(X=k) = \mathcal{B}_{2,p}(k)$ for $k \in
\{0,1,2\}$ and $\mathbf{P}(X=k) = 0$ otherwise. This probability
distribution reflects the expectation for connectivity of neuron pairs
in anisotropic networks. A numeric analysis of the anisotropic sample
graphs reveals that bidirectionally connected pairs appear almost
twice as often as expected from the overall connection
probability ($p=0.116$) and equations \ref{eq:pairs}, similarly as
reported by Song et al.~(\autoref{fig:two_neuron_probs} A). However,
comparing pair probabilities in anisotropic networks with the
probabilities in their rewired counterparts we find that anisotropy is
does not influence the occurrence of two-neuron motifs
(\autoref{fig:two_neuron_probs} B) In fact, expected connections in
neuron pairs are identical in distance-dependent and rewired
anisotropic networks (\autoref{suppfig:two_neuron_dist_rew}).

\begin{figure}[ht]
  \centering
  \makebox{%
    \begin{overpic}[height=0.17\textheight]{%
        plots/c5f1462b_aniso_rand.pdf}
      \put(13.1,64.8){\small\textbf{A}}
      %\put(14.3,78.9){\small\textbf{A}}
    \end{overpic}
    \hspace{0.45cm}
    \begin{overpic}[height=0.17\textheight]{%
        plots/c5f1462b_aniso_rew.pdf}      
      \put(15.8,64.8){\small\textbf{B}}
    \end{overpic}
  }%
  \captionsetup{skip=10pt}
  \caption{\textbf{Overrepresentation of reciprocal connections in
      anisotropic networks due to distance-dependent connectivity}
    Extracting the counts of unconnected, one-directionally and
    bidirectionally connected neuron pairs in the anisotropic sample
    graphs, overrepresentation of reciprocally connected pairs is
    identified as a feature of the network's distance dependency as
    opposed to anisotropy in connectivity. \textbf{A)} Showing the
    quotient of the counts for the three pair types, extracted from the
    set of sample graphs, with the number of expected pairs in Gilbert
    random graphs $G(n,p)$, where $n=1000$ and $p=0.116$ were matched
    to the sample graph parameters. While single connections appear
    less often than in Gilbert random graphs, reciprocal connections
    are significantly overrepresented. Errorbars SEM. \textbf{B)}
    Comparing appearance of connection pairs in the anisotropic sample
    graphs with their respective appearance in the rewired sample
    graphs, we find that eliminating anisotropy does not significantly
    change the counts for the connection types, indicating that
    anisotropy does not influence two neuron connection
    probabilities. Errorbars SEM.  (\smtcite{c5f1462b})}
  \label{fig:two_neuron_probs}
\end{figure}  


We further support this observation by computing the probability
distribution for the expected number of edges between to random
vertices in the anisotropic graph model. For this we assume that only
the distance-dependent connection probability $C(x)$ determines the
occurrence of edges in vertex pairs in the anisotropic graph
model. Then, using the probability distribution $f(x)$ for the a
random neuron pair to be at distance $x$, we calculate
\begin{align*}
\mathbf{P}(X=0) & = \int_0^{\sqrt{2}} (1-C(x))^2 f(x)\,
dx, \\
\mathbf{P}(X=1) & = \int_0^{\sqrt{2}} 2 C(x) (1-C(x)) f(x) \, dx \quad \mathrm{and}\\
\mathbf{P}(X=2) & = \int_0^{\sqrt{2}} C(x)^2 f(x) \, dx. 
\end{align*}
Inserting the distance-dependent connection probabilities $C(x)$ in
the anisotropic graph model as computed in
Theorem~\ref{theorem:distance_prof} and the probability distribution
$f(x)$ from Theorem~\ref{theorem:distance_square} we obtain
\begin{align*} 
\mathbf{P}(X=0) & = 0.791336 && 0.7907  \pm 0.0008\\
\mathbf{P}(X=1) & = 0.184151 && 0.1846  \pm 0.0007\\
\mathbf{P}(X=2) & = 0.024513 && 0.02462  \pm 0.00009,
\end{align*}
perfectly matching the probabilities extracted from anisotropic sample
graphs in the right column (error SEM, \smtcite{c5f1462b}). Noting
that distance-dependency alone is sufficient to accurately predict
edge probabilities in neuron pairs in the anisotropic network model
and combined with the observations in \autoref{fig:two_neuron_probs},
we conclude that varying degrees of anisotropy do not affect the
occurrence of neuron pair motifs.


% From Python c5f1462b
% Unconn:       0.790732332332  +- 0.000817274958369
% Single:       0.184638238238  +- 0.00074516663005
% Recip:        0.0246294294294  +- 8.63947164473e-05

% Mathematica (expected_two_neuron.nb) (rounded)
% Unconn:       0.791336
% Single:       0.184151
% Recip:        0.024513 



Song and Perin report that distance-dependency is not the cause for
overrepresentation. To test this tuned networks!



%%% Local Variables: 
%%% mode: latex
%%% TeX-master: "../dplths_document"
%%% End: 
